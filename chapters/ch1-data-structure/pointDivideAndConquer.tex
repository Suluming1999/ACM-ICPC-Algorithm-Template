\section{点分治}
\textbf{问题场景}
~\\
\par \noindent 点分治是大规模处理树上路径问题的工具。大意是找到一个点,递归统计其所有子树的答案,然后利用容斥原理或其它方式合并答案,最后得到整棵树的答案。
~\\
\par \noindent \textbf{树重心性质}(加粗的两句话互为充要条件,常用来找重心)

\begin{enumerate}
\item \textbf{对于一棵树 n 个节点的无根树,找到一个点,使得把树变成以该点为根的有根树时,最大子树的结点数最小。}
\item \textbf{对于一个大小为 n 的树,删去重心及与它关联的边后,分裂出的所有子树的大小均不超过 n/2。对于包含至少一个结点的树,它的重心只可能有 1 或 2 个。}
\item 树中所有点到某个点的距离和中,到重心的距离和是最小的,如果有两个重心,他们的距离和一样。
\item 把两棵树通过一条边相连,新的树的重心在原来两棵树重心的连线上。
\item 一棵树添加或者删除一个叶节点,树的重心最多只移动一条边的位置。
\item 一棵树最多有两个重心,且相邻。
\end{enumerate}
~\\

\textbf{点分治步骤}
~\\
\begin{enumerate}
\item 找到整棵树的重心点 $rt$,由 $rt$ 向下递归求解。
\item 统计以 $rt$ 为根的子树的答案 $ans_{rt}^\prime$,并使用容斥原理、染色法等去除不合法的答案,得到 $rt$ 的最终答案 $ans_{rt}$
\item 对 $rt$ 的子树 $ch_i$ 求解,同样先找到以 $ch_i$ 为根的子树的重心 $r^\prime$,然后从重心 $r^\prime$ 向下递归求解,回到步骤 (1).
\end{enumerate}

\begin{minted}{c++}
#include <bits/stdc++.h>

using namespace std;
const int N = 10010;
int h[N], e[2 * N], ne[2 * N], w[2 * N], idx;
int n, K;
void add(int a, int b, int c) {
    e[idx] = b, ne[idx] = h[a], w[idx] = c, h[a] = idx++;
}
int rt, sz[N];
bool del[N]; // 该点是否被删掉
void dfs_rt(int u, int fa, int tot) {
    sz[u] = 1;
    int mx = 0;
    for (int i = h[u]; i != -1; i = ne[i]) {
        int v = e[i];
        if (v == fa || del[v])
            continue;
        dfs_rt(v, u, tot);
        sz[u] += sz[v];
        mx = max(mx, sz[v]);
    }
    mx = max(mx, tot - sz[u]);
    if (2 * mx <= tot)
        rt = u;
}
void dfs_sz(int u, int fa) {
    sz[u] = 1;
    for (int i = h[u]; i != -1; i = ne[i]) {
        int v = e[i];
        if (v == fa || del[v])
            continue;
        dfs_sz(v, u);
        sz[u] += sz[v];
    }
}
int bit[5000010];
int lowbit(int x) {return x & -x;}
void update(int k, int x) {
    if (!k) return bit[k] += x, void;
    for (; k <= K; k += lowbit(k)) bit[k] += x;
}
int query(int k) {
    if (k < 0) return 0;
    int ret = bit[0];
    for (; k; k -= lowbit(k))ret += bit[k];return ret;
}
int cnt, d[N];
void dfs_dist(int u, int fa, int dist) {
    d[++cnt] = dist;
    for (int i = h[u]; i != -1; i = ne[i]) {
        int v = e[i];
        if (v == fa || del[v]) continue;
        dfs_dist(v, u, dist + w[i]);
    }
}
void dfs_clear(int u, int fa, int dist) {
    update(dist, -1);
    for (int i = h[u]; i != -1; i = ne[i]) {
        int v = e[i];
        if (v == fa || del[v]) ontinue;
        dfs_clear(v, u, dist + w[i]);
    }
}
int work(int u, int tot) {
    int ans = 0;
    dfs_rt(u, 0, tot);
    u = rt;
    dfs_sz(u, 0);
    del[u] = 1;
    update(0, 1); // 根节点距离是0

    for (int i = h[u]; i != -1; i = ne[i]) {
        int v = e[i];
        if (del[v]) continue;
        cnt = 0;
        dfs_dist(v, u, w[i]);
        for (int k = 1; k <= cnt; k++) ans += query(K - d[k]);
        for (int k = 1; k <= cnt; k++) update(d[k], 1);
    }
    dfs_clear(u, 0, 0);
    for (int i = h[u]; i != -1; i = ne[i]) {
        int v = e[i];
        if (del[v])
            continue;
        ans += work(v, sz[v]);
    }
    return ans;
}
int main() {
    memset(h, -1, sizeof h);
    idx = 0;
    memset(del, 0, sizeof del);

    for (int i = 1; i < n; i++) {
        int a, b, c;
        cin >> a >> b >> c;
        ++a, ++b;
        add(a, b, c);
        add(b, a, c);
    }

    cout << work(1, n) << '\n';
}
\end{minted}
