\section{树套树}

\subsection{树状数组套权值线段树}

\par \noindent 二维数颜色

\begin{minted}{c++}
#include <bits/stdc++.h>
using namespace std;
template <class T = int> T rd() {
    T res = 0;
    char ch = getchar();
    while (!isdigit(ch)) ch = getchar();
    while (isdigit(ch)) res = (res << 1) + (res << 3) + (ch ^ 48), ch = getchar();
    return res;
}
const int N = 100010;
int a[N], n, m, ans[N], last[N];
struct nodeq {
    int l, L, R, id;
};
vector<nodeq> q[N];
struct node {
    int l, r, v;
} tree[N * 200];
int rt[N], cnt, lim;
void update(int &u, int l, int r, int pos, int v) {
    if (!u)
        u = ++cnt;

    tree[u].v += v;

    if (l == r)
        return;

    int mid = l + r >> 1;

    if (pos <= mid)
        update(tree[u].l, l, mid, pos, v);

    if (pos > mid)
        update(tree[u].r, mid + 1, r, pos, v);

    //tree[u].v=tree[tree[u].l].v+tree[tree[u].r].v;
}
int query(int u, int l, int r, int L, int R) {
    if (!u)
        return 0;

    if (L <= l && r <= R) return tree[u].v;

    int mid = l + r >> 1;
    int v = 0;

    if (L <= mid) v += query(tree[u].l, l, mid, L, R);
    if (R > mid) v += query(tree[u].r, mid + 1, r, L, R);
    return v;
}
/* 树状数组 */
// rt[i] 保存每个单点对应的权值线段树根节点,
int lowbit(int x) { return x & -x; }
void add(int k, int pos, int v) {
    for (; k <= n; k += lowbit(k)) update(rt[k], 0, lim, pos, v);
}
int ask(int k, int L, int R) {
    int ans = 0;
    for (; k; k -= lowbit(k)) ans += query(rt[k], 0, lim, L, R);
    return ans;
}
int main() {
    n = rd(), m = rd();

    for (int i = 1; i <= n; i++) a[i] = rd();

    lim = *max_element(a + 1, a + 1 + n);

    for (int i = 1; i <= m; i++) {
        int x0 = rd(), y0 = rd(), x1 = rd(), y1 = rd();
        q[x1].push_back({x0, y0, y1, i});
    }
    for (int i = 1; i <= n; i++) {
        if (last[a[i]]) add(last[a[i]], a[i], -1);

        add(i, a[i], 1);
        last[a[i]] = i;
        for (auto t : q[i]) ans[t.id] = ask(i, t.L, t.R) - ask(t.l - 1, t.L, t.R);
    }
    for (int i = 1; i <= m; i++) printf("%d\n", ans[i]);

}
\end{minted}

\subsection{下标线段树套(值域)平衡树}
\par \noindent 首先,对于维护值域的平衡树通常可以采用\textbf{权值线段树}代替。
~\\
\par \noindent 但是一般情况下,平衡树维护值域时,下标信息就会丢失,意味着不能维护类似下表是第 $i$ 个数是谁(除非开数组记录),因此也就不能维护区间 $[l,r]$ 内某个数的排名前驱后继等操作。
~\\
\par \noindent 想要维护同时维护下标,必须外层用一个数据结构维护下标,比如可以采用(下标)线段树套(值域)平衡树。此时便可以支持同时维护【下标】和【值域】的操作。
~\\
\begin{enumerate}
\item \textbf{查询 $k$ 在区间 $[l,r]$ 的排名:}找到下标线段树上 $[l,r]$ 所对应的 $\log N$个终止结点,统计每个结点平衡树内 $<k$元素个数,求和 $+1$ 就是 $k$ 的排名,复杂度$O(n\log^2n)$。
\item \textbf{查询区间 $[l,r]$ 内排名为 $k$ 的值:}二分答案然后用Getrank判断,复杂度$O(n\log^3n)$。
\item \textbf{修改某一位值上的数值 $a[i]=x$:}只要在所有包含 $a[i]$ 的平衡树($\log N$个)中删除 $a[i]$,然后再插入 $x$ 即可,复杂度$O(n\log^2n)$。
\item \textbf{查询 $x$ 在区间内 $[l,r]$ 的前驱:}在 $[l,r]$ 的所有终止结点的平衡树内查询前驱,取所有中止节点前驱的最大值,复杂度$O(n\log^2n)$。
\item \textbf{查询 $x$ 在区间内 $[l,r]$ 的后继:}在 $[l,r]$ 的所有终止结点的平衡树内查询后继,取所有中止节点后继的最小值,复杂度 $O(n\log^2n)$。
\end{enumerate}

\begin{minted}{c++}
#include<bits/stdc++.h>

using namespace std;
const int N = 50010;
const int seed = []() {
    random_device rds;
    return rds();
}();
mt19937 rd(seed);
// Treap 维护值域
struct T {
        const int key;      // 随机值 满足堆的性质
        int ls, rs, sz;     // 基本值数量  该题没用
        int val;            // 维护序列的值
        T() : key(rd()) {
            ls = rs = 0;
            sz = 0;
            val = 0;
        }
}t[N * 40];
int cnt;
// 无旋Treap平衡树 一个Treap只需要知道根节点是谁即可
struct Treap {
    int rt;
    int newnode(int x) {
        t[++cnt].val = x;
        t[cnt].sz = 1;
        return cnt;
    }
    void pushup(int u) {
        t[u].sz = t[t[u].ls].sz + t[t[u].rs].sz + 1;
    }
    // 两棵树合并
    int merge(int x, int y) {
        if (!x || !y) {
            return x | y;
        }
        int u = 0;
        if (t[x].key < t[y].key) {
            u = x, t[u].rs = merge(t[u].rs, y);
        } else {
            u = y, t[u].ls = merge(x, t[u].ls);
        }
        pushup(u);
        return u;
    }
    // 按val分裂 分裂出 <= val 的部分
    void split(int u, int val, int &x, int &y) {
        if (!u) {
            x = y = 0;
            return;
        }
        if (t[u].val <= val) {
            x = u;
            split(t[u].rs, val, t[x].rs, y);
        }
        else {
            y = u;
            split(t[u].ls, val, x, t[y].ls);
        }
        pushup(u);
    }
    // 分裂出 <= val 的部分
    pair<int, int> split(int u, int val) {
        int x, y;
        split(u, val, x, y);
        return make_pair(x, y);
    }
    void ins(int val) {
        auto [x, y] = split(rt, val);
        rt = merge(merge(x, newnode(val)), y);
    }
    void del(int val) {
        auto [w, z] = split(rt, val);
        auto [x, y] = split(w, val - 1);
        y = merge(t[y].ls, t[y].rs); // 删去一个点相当于合并
        rt = merge(merge(x, y), z);
    }
    // val的排名 = 小于val的个数 + 1
    int getrank(int val) {
        auto [x, y] = split(rt, val - 1);
        int cnt = t[x].sz + 1;
        rt = merge(x, y);
        return cnt;
    }
    // 排名时 k 的值
    int getval(int k) {
        int u = rt;
        while(u) {
            if(t[t[u].ls].sz + 1 == k) break;
            else if(t[t[u].ls].sz >= k) u = t[u].ls;
            else {
                k -= t[t[u].ls].sz + 1;
                u = t[u].rs;
            }
        }
        return t[u].val;
    }
    int pre(int val) {
        auto [x, y] = split(rt, val - 1);
        int u = x;
        while(t[u].rs) u = t[u].rs;
        rt = merge(x, y);
        return u ? t[u].val : -2147483647;
    }
    int suc(int val) {
        auto [x, y] = split(rt, val);
        int u = y;
        while(t[u].ls) u = t[u].ls;
        rt = merge(x, y);
        return u ? t[u].val : 2147483647;
    }
};
int n, m;
int a[N];
// 线段树
struct Node {
    int l, r;
    Treap treap;
}tree[N << 2];

void build(int u, int l, int r) {
    tree[u].l = l, tree[u].r = r;

    for (int i = l; i <= r; i++) tree[u].treap.ins(a[i]);

    if (l == r) return;

    int mid = l + r >> 1;
    build(u << 1, l, mid); build(u << 1 | 1, mid + 1, r);
}
void modify(int u, int pos, int x) {
    tree[u].treap.del(a[pos]);
    tree[u].treap.ins(x);

    if (tree[u].l == tree[u].r)
        return;

    int mid = tree[u].l + tree[u].r >> 1;

    if (pos <= mid)
        modify(u << 1, pos, x);
    else
        modify(u << 1 | 1, pos, x);
}
int Getrank(int u, int l, int r, int val) {
    if (l <= tree[u].l && tree[u].r <= r)
        return tree[u].treap.getrank(val) - 1;

    int mid = tree[u].l + tree[u].r >> 1;
    int k = 0;

    if (l <= mid)
        k += Getrank(u << 1, l, r, val);

    if (r > mid)
        k += Getrank(u << 1 | 1, l, r, val);

    return k;
}
int Getval(int u, int l, int r, int k) {
    int vl = 0, vr = 1e8;

    while (vl < vr) {
        int mid = vl + vr + 1 >> 1;

        if (Getrank(u, l, r, mid) + 1 <= k)
            vl = mid; //注意二分 需要二分排名<=k的最大数(数可能重复)
        else
            vr = mid - 1;
    }

    return vl;
}
int Getpre(int u, int l, int r, int val) {
    if (tree[u].l >= l && tree[u].r <= r)
        return tree[u].treap.pre(val);

    int mid = tree[u].l + tree[u].r >> 1;

    if (r <= mid)
        return Getpre(u << 1, l, r, val);
    else if (l > mid)
        return Getpre(u << 1 | 1, l, r, val);
    else
        return max(Getpre(u << 1, l, r, val), Getpre(u << 1 | 1, l, r, val));
}
int Getsuc(int u, int l, int r, int val) {
    if (tree[u].l >= l && tree[u].r <= r)
        return tree[u].treap.suc(val);

    int mid = tree[u].l + tree[u].r >> 1;

    if (r <= mid)
        return Getsuc(u << 1, l, r, val);
    else if (l > mid)
        return Getsuc(u << 1 | 1, l, r, val);
    else
        return min(Getsuc(u << 1, l, r, val), Getsuc(u << 1 | 1, l, r, val));
}

int main() {
    ios::sync_with_stdio(false);cin.tie(nullptr);cout.tie(nullptr);
    cin >> n >> m;

    for (int i = 1; i <= n; i++) cin >> a[i];

    build(1, 1, n);

    while (m--) {
        int op, l, r, k;
        cin >> op;
    
        if (op == 1) {
            cin >> l >> r >> k;
            cout << Getrank(1, l, r, k) + 1 << '\n';
        } else if (op == 2) {
            cin >> l >> r >> k;
            cout << Getval(1, l, r, k) << '\n';
        } else if (op == 3) {
            cin >> l >> k;
            modify(1, l, k);
            a[l] = k;
        } else if (op == 4) {
            cin >> l >> r >> k;
            cout << Getpre(1, l, r, k) << '\n';
        } else {
            cin >> l >> r >> k;
            cout << Getsuc(1, l, r, k) << '\n';
        }
    }
    return 0;
}
\end{minted}


\subsection{值域线段树套(下标)平衡树}

\par \noindent 当然可以用\textbf{动态开点}线段树维护值域,也就是权值线段树,用平衡树维护下标
~\\
\begin{enumerate}
\item \textbf{查询 $k$ 在区间 $[l,r]$ 的排名:} 找到值域线段树上 $[0,k-1]$ 所对应的 $\log N$个终止结点,统计每个结点平衡树下标在 $[l,r]$ 区间内元素个数即求出区间为 $[l,r]$ 内值域在 $[0,k-1]$点的个数,求和+1就是 $k$ 的排名,复杂度 $O(n\log^2n)$。

\item \textbf{查询区间 $[l,r]$ 内排名为 $k$ 的值:}在值域线段树上二分,每次看值域范围 $[vl,mid]$ 中 $[L,R]$ 区间中数的个数 = tree[u].treap.query(L, R) 与 $k$ 的关系,$\ge k$ 递归左子树,否则递归右子树,复杂度 $O(n\log^2 n)$。
\item \textbf{修改某一位值上的数值 $a[i]=x$:}只要在所有包含 $a[i]$ 的平衡树($\log {|a[i]|}$个) 中删除 $a[i]$,然后再插入 $x$ 即可,复杂度 $O(n\log^2n)$。
\item \textbf{查询 $x$ 在区间内$[l,r]$ 的前驱:} 先用操作1查询 $x$ 在 $[l,r]$ 内的排名 $k$,然后再用操作2查询排名为$k-1$ 数,复杂度 $O(n\log^2n)$。
\item \textbf{查询 $x$ 在区间内 $[l,r]$ 的后继:}先用Query求出区间在 $[l,r]$ 内值域 $[0,x]$ 的个数 $k$,然后再用操作2查询排名为 $k+1$数,复杂度 $O(n\log^2n)$。
\end{enumerate}

\begin{minted}{c++}
#include <random>
#include <iostream>
#include <algorithm>
using namespace std;
const int N = 100010;
const int MINF = -2147483647, MAXF = 2147483647;
const int seed = []() {
    random_device rds;
    return rds();
}();
mt19937 rd(seed);

int n, m;
int a[N];
int Root;
// Treap 维护值域
struct T {
        const int key;      // 随机值 满足堆的性质
        int ls, rs, sz;     // 基本值数量 
        int val;            // 维护序列的值
        T() : key(rd()) {
            ls = rs = 0;
            sz = 0;
            val = 0;
        }
}t[N * 40];
int cnt;

// 无旋Treap平衡树 一个Treap只需要知道根节点是谁即可
struct Treap {
    int rt;
    int newnode(int x) {
        t[++cnt].val = x;
        t[cnt].sz = 1;
        return cnt;
    }
    void pushup(int u) {
        t[u].sz = t[t[u].ls].sz + t[t[u].rs].sz + 1;
    }
    // 两棵树合并
    int merge(int x, int y) {
        if (!x || !y) {
            return x | y;
        }
        int u = 0;
        if (t[x].key < t[y].key) {
            u = x, t[u].rs = merge(t[u].rs, y);
        } else {
            u = y, t[u].ls = merge(x, t[u].ls);
        }
        pushup(u);
        return u;
    }
    // 按val分裂 分裂出 <= val 的部分
    void split(int u, int val, int &x, int &y) {
        if (!u) {
            x = y = 0;
            return;
        }
        if (t[u].val <= val) {
            x = u;
            split(t[u].rs, val, t[x].rs, y);
        }
        else {
            y = u;
            split(t[u].ls, val, x, t[y].ls);
        }
        pushup(u);
    }
    // 分裂出 <= val 的部分
    pair<int, int> split(int u, int val) {
        int x, y;
        split(u, val, x, y);
        return make_pair(x, y);
    }
    void ins(int val) {
        auto [x, y] = split(rt, val);
        rt = merge(merge(x, newnode(val)), y);
    }
    void del(int val) {
        auto [w, z] = split(rt, val);
        auto [x, y] = split(w, val - 1);
        y = merge(t[y].ls, t[y].rs); // 删去一个点相当于合并
        rt = merge(merge(x, y), z);
    }
    // val的排名 = 小于val的个数 + 1
    int getrank(int val) {
        auto [x, y] = split(rt, val - 1);
        int cnt = t[x].sz + 1;
        rt = merge(x, y);
        return cnt;
    }
    // 排名时 k 的值
    int getval(int k) {
        int u = rt;
        while(u) {
            if(t[t[u].ls].sz + 1 == k) break;
            else if(t[t[u].ls].sz >= k) u = t[u].ls;
            else {
                k -= t[t[u].ls].sz + 1;
                u = t[u].rs;
            }
        }
        return t[u].val;
    }
    int pre(int val) {
        auto [x, y] = split(rt, val - 1);
        int u = x;
        while(t[u].rs) u = t[u].rs;
        rt = merge(x, y);
        return u ? t[u].val : -2147483647;
    }
    int suc(int val) {
        auto [x, y] = split(rt, val);
        int u = y;
        while(t[u].ls) u = t[u].ls;
        rt = merge(x, y);
        return u ? t[u].val : 2147483647;
    }
    // [L, R]点的个数
    int query(int L, int R) {
        auto [x, y] = split(rt, L - 1);
        auto [z, w] = split(y, R);
        int res = t[z].sz;
        rt = merge(merge(x, z), w);
        return res;
    }
};
// 动态开点权值线段树
struct Node {
    int l, r;
    Treap treap;
} tree[N * 40];
int idx;
void Ins(int &u, int l, int r, int pos, int x) {
    if (!u) u = ++idx;

    tree[u].treap.ins(x);

    if (l == r) return;
    int mid = l + r >> 1;

    if (pos <= mid)
        Ins(tree[u].l, l, mid, pos, x);
    else
        Ins(tree[u].r, mid + 1, r, pos, x);
}
void Del(int u, int l, int r, int pos, int x) {
    if (!u) return;
    tree[u].treap.del(x);
    
    if (l == r) return;
    int mid = l + r >> 1;
    if (pos <= mid) Del(tree[u].l, l, mid, pos, x);
    else Del(tree[u].r, mid + 1, r, pos, x);
}
// 实际上是求在区间[L, R]内值域在[vl, vr]点的个数 二维数点
int Query(int u, int l, int r, int vl, int vr, int L, int R) {
    if (!u) return 0;

    if (vl <= l && r <= vr) return tree[u].treap.query(L, R);

    int mid = l + r >> 1;
    int v = 0;
    if (vl <= mid) v += Query(tree[u].l, l, mid, vl, vr, L, R);
    if (vr > mid) v += Query(tree[u].r, mid + 1, r, vl, vr, L, R);
    return v;
}
// 值域线段树二分
int Getval(int u, int l, int r, int L, int R, int k) {
    if (l == r) return l;
    int mid = l + r >> 1;
    int tmp = 0;
    
    if (tree[u].l)  // 查询(下标)平衡树 [L, R] 数的个数
            tmp = tree[tree[u].l].treap.query(L, R);
    if (tmp >= k)
        return Getval(tree[u].l, l, mid, L, R, k);
    else
        return Getval(tree[u].r, mid + 1, r, L, R, k - tmp);
}
int main() {

    cin >> n >> m;
    for (int i = 1; i <= n; i++) {
        cin >> a[i];
        Ins(Root, 0, 1e8, a[i], i); // 在值为a[i]的位置插入下标i
    }

    while (m--) {
        int op, l, r, k;
        cin >> op;

        if (op == 1) {
            cin >> l >> r >> k;
            cout << Query(Root, 0, 1e8, 0, k - 1, l, r) + 1 << '\n';
        } else if (op == 2) {
            cin >> l >> r >> k;
            cout << Getval(Root, 0, 1e8, l, r, k) << '\n';
        } else if (op == 3) {
            cin >> l >> k;
            Del(Root, 0, 1e8, a[l], l);
            Ins(Root, 0, 1e8, k, l);
            a[l] = k;
        } else if (op == 4) {
            cin >> l >> r >> k;
            int x = Query(Root, 0, 1e8, 0, k - 1, l, r) + 1;

            if (x <= 1)
                cout << MINF << '\n';
            else
                cout << Getval(Root, 0, 1e8, l, r, x - 1) << '\n';
        } else {
            cin >> l >> r >> k;
            int x = Query(Root, 0, 1e8, 0, k, l, r);
            if (x == r - l + 1)
                cout << MAXF << '\n';
            else
                cout << Getval(Root, 0, 1e8, l, r, x + 1) << '\n';
        }
    }
    return 0;
}
\end{minted}