\section{GCD与exGCD}
\par \noindent 求解 $gcd(a,b)$
\begin{minted}{c++}
int gcd(int a, int b) {
    return b ? gcd(b, a % b) : a;
}
\end{minted}

\par \noindent 求解一组特解 $x_0,y_0$ 使得 $ax+by = gcd(a,b),d=gcd(a,b)$ ,其通解为
$$
x_0+k×\frac{b}{d},y_0-k×\frac{a}{d}, k \in Zx_0+k×\frac{b}{d},y_0-k×\frac{a}{d}, k \in Z
$$

\begin{minted}{c++}
int exgcd(int a, int b, int &x, int &y) {
    if (!b) {
        x = 1; y = 0;
        return a;
    }
    int d = exgcd(b, a % b, y, x);
    y -= a / b * x;
    return d;
}
\end{minted}

\section{一阶同余方程}

$$
ax+by = c
$$

\par \noindent 方程有解当且仅当 $d = gcd(a,b),d\ |\ c$ ,由exGCD先解得 $x_0,y_0$ ,得特解 $x_1 = x_0*c/d,y_1=y_0*c/d$,通解仍为
$$
x_1 + k*\frac{b}{d},y_1-k*\frac{a}{d},k\in Z
$$

$$
ax \equiv b\ (mod\ m)​
$$

若$gcd(a,m) = 1$,则$a$存在逆元,求逆即可;否则转化为 $ax + my = b$ 求解
\section{线性逆元}
\begin{minted}{c++}
inv[0] = inv[1] = 1;
for (int i = 2; i <= n; ++i) {
    inv[i] = (mod - mod / i) * inv[mod % i] % mod;
}
\end{minted}

\section{线性筛素数 / 积性函数}

\par \noindent 质数 $pri$ ,欧拉函数 $phi$ ,莫比乌斯函数 $mob$ ,约数个数 $d$ ,约数和 $s$ .辅助函数
~\\
\par \noindent $num(x):x$ 最小质因子的幂次
~\\
\par \noindent $sp(x):x$ 最小质因子的等比数列和 $\left(1+p^1+p^2+\ldots+p^n\right)$
\begin{minted}{c++}
int n, pri[N], phi[N], mob[N], d[N], num[N], s[N], sp[N], cnt;
bool np[N];
void init() {
    np[0] = np[1] = true;
    mob[1] = 1; phi[1] = 1; d[1] = 1;
    for (int i = 2; i <= n; ++i) {
        if (!np[i]) {
            pri[++cnt] = i;
            phi[i] = i - 1; mob[i] = -1;
            d[i] = 2; num[i] = 1;
            s[i] = i + 1; sp[i] = i + 1;
        }
        for (int j = 1; j <= cnt && i * pri[j] <= n; ++j) {
            np[i * pri[j]] = true;
            if (i % pri[j] == 0) {
                phi[i * pri[j]] = phi[i] * pri[j];
                mob[i * pri[j]] = 0;
                d[i * pri[j]] = d[i] / (num[i] + 1) * (num[i] + 2); num[i * pri[j]] = num[i] + 1;
                s[i * pri[j]] = s[i] / sp[i] * (sp[i] * pri[j] + 1); sp[i * pri[j]] = sp[i] * pri[j] + 1;
                break;
            }
            phi[i * pri[j]] = phi[i] * (pri[j] - 1);
            mob[i * pri[j]] = -mob[i];
            d[i * pri[j]] = d[i] * 2; num[i * pri[j]] = 1;
            s[i * pri[j]] = s[i] * (pri[j] + 1); sp[i * pri[j]] = pri[j] + 1;
        }
    }
}
\end{minted}
\section{杜教筛——phi与mob前缀和}
\par \noindent 约数个数 $d$ 和约数和 $s$ 的前缀和均可用整除分块计算。
\begin{minted}{c++}
#include<bits/stdc++.h>
using namespace std;
typedef long long ll;
const int N = 1e7;
bool np[N + 5];
vector <int> pri;
int mob[N + 5], smob[N + 5];
ll phi[N + 5], sphi[N + 5];
unordered_map <ll, int> usmob;
unordered_map <ll, ll> usphi;
int read() {
    char ch = getchar();
    int re = 0;
    while (ch < '0' || ch>'9') ch = getchar();
    while (ch >= '0' && ch <= '9') { re = (re << 1) + (re << 3) + ch - '0'; ch = getchar(); }
    return re;
}
void init() {
    mob[1] = 1; phi[1] = 1;
    for (int i = 2; i <= N; ++i) {
        if (!np[i]) {
            pri.push_back(i);
            mob[i] = -1; phi[i] = i - 1;
        }
        for (int p : pri) {
            if (p * i > N) break;
            np[p * i] = 1;
            if (i % p == 0) {
                mob[p * i] = 0;
                phi[p * i] = phi[i] * p;
                break;
            }
            mob[p * i] = mob[p] * mob[i];
            phi[p * i] = phi[p] * phi[i];
        }
    }
    for (int i = 1; i <= N; ++i) {
        smob[i] = smob[i - 1] + mob[i];
        sphi[i] = sphi[i - 1] + phi[i];
    }
}
ll sum_mob(ll x) {
    if (x <= N)  return smob[x];
    if (usmob.count(x))  return usmob[x];
    ll ans = 1;
    for (ll l = 2, r; l <= x; l = r + 1) {
        r = x / (x / l);
        ans -= (r - l + 1) * sum_mob(x / l);
    }
    usmob[x] = ans;
    return ans;
}
ll sum_phi(ll x) {
    if (x <= N)  return sphi[x];
    if (usphi.count(x))  return usphi[x];
    ll ans = x * (x + 1) / 2;
    for (ll l = 2, r; l <= x; l = r + 1) {
        r = x / (x / l);
        ans -= (r - l + 1) * sum_phi(x / l);
    }
    usphi[x] = ans;
    return ans;
}
int main() {
    int T, tmp;
    T = read();
    init();
    while (T--) {
        tmp = read();
        printf("%lld %lld\n", sum_phi(N), sum_mob(N));
    }
    return 0;
}
\end{minted}

\section{O(n)预处理与O(log(n))分解质因数}
\begin{minted}{c++}
#include<bits/stdc++.h>
using namespace std;
const int N = 1e7 + 5;
int lim = 1e7;
int fac[N], pri[N]; //fac(i) 表示i的除1以外的最小质因子
bool np[N];
void init() {
    int cnt = 0;
    np[1] = 1; fac[1] = 1;
    for (int i = 2; i <= lim; ++i) {
        if (!np[i]) { pri[++cnt] = i; fac[i] = i; }
        for (int j = 1; j <= cnt && i * pri[j] <= lim; ++j) {
            np[i * pri[j]] = 1; fac[i * pri[j]] = pri[j];
            if (i % pri[j] == 0) break;
        }
    }
}
int main() {
    int T, n;
    cin >> T;
    init();
    while (T--) {
        int cnt = 0;
        cin >> n;
        while (n > 1) {
            printf("%d ", fac[n]);
            n /= fac[n];
        }
        printf("\n");
    }
}
\end{minted}

\section{扩展欧拉定理}
$$
a^b\equiv \begin{cases} a^{b\ \bmod\ \varphi(p)},&\gcd(a,p)=1\\ a^b,&\gcd(a,p)\ne1,b<\varphi(p)\\ a^{b\ \bmod\ \varphi(p)\ +\ \varphi(p)},&\gcd(a,p)\ne1,b\ge\varphi(p) \end{cases} \pmod p
$$

\section{中国剩余定理(Chinese Remainder Theorem)}
\par \noindent 设整数 $m_1,m_2,...,m_n$ 其中 **任两数** 互质,$M = \prod_{i = 1}^nm_i,\ M_i = \frac{M}{m_i}$,则对任意正整数 $a_1, a_2, \dots, a_n$,线性同余方程组
$$
\begin{cases} 
    x &\equiv a_1 \pmod {m_1} \\ x &\equiv a_2 \pmod {m_2} \\ &\vdots \\ x &\equiv a_n \pmod {m_n} 
\end{cases}
$$
\par \noindent 在模 $M$ 意义下有唯一解 $x \equiv \sum_{i = 1}^{n} a_iM_iv_i \pmod M$,其中 $v_i \equiv {M_i}^{-1} \pmod {m_i}$
~\\
\par \noindent 如果没有 $m_1,m_2,...,m_n$任两数互质的条件,则需使用扩展中国剩余定理(exCRT).
~\\
\par \noindent 给出exCRT的代码实现(两两合并 + exGCD求解).
\begin{minted}{c++}
#include<bits/stdc++.h>
using namespace std;
typedef long long ll;
typedef __int128 lll;
const int N = 1e5 + 5;
int n;
ll a[N], m[N];
ll exgcd(ll a, ll b, ll &x, ll &y) {
    if (b == 0) {
        x = 1; y = 0;
        return a;
    }
    ll re = exgcd(b, a % b, x, y), tmp;
    tmp = x; x = y; y = tmp - (a / b) * y;
    return re;
}
ll exCRT(int n, ll *a, ll *m) {
    ll ans = a[1], mod = m[1];
    for (int i = 2; i <= n; ++i) { //两两合并
        ll x, y, d = exgcd(mod, m[i], x, y), tmp;
        if ((a[i] - ans) % d) return -1;

        lll px = (lll)(a[i] - ans) / d * x;
        tmp = m[i] / d;
        x = (px % tmp + tmp) % tmp;
        ans = ((lll)x * mod + ans) % (mod / d * m[i]);
        mod = mod / d * m[i];
    }
    return ans;
}
int main() {
    cin >> n;
    for (int i = 1; i <= n; ++i) cin >> m[i] >> a[i];
    ll prt = exCRT(n, a, m);
    if (prt == -1) printf("No solution.\n");
    else printf("%lld", prt);
    return 0;
}
\end{minted}

\section{Miller-Rabin 素性测试与Pollard-rho质因数分解}

\par \noindent 数据组数 $T$,求解 $n$ 的最大质因数. (若求所有质因数稍加修改即可)
\begin{minted}{c++}
#include<bits/stdc++.h>
using namespace std;
typedef long long ll;
ll T, maxm;
ll qpow(ll x, ll exp, ll mod) {
    ll re = 1;
    while (exp) {
        if (exp & 1) re = (__int128)re * x % mod;
        x = (__int128)x * x % mod;
        exp >>= 1;
    }
    return re;
}
bool is_prime(ll p) {
    if (p < 3) return p == 2;
    if (!(p & 1)) return false;
    const static ll bse[] = {2, 325, 9375, 28178, 450775, 9780504, 1795265022};
    ll d = p - 1, r = 0;
    while (!(d & 1)) { d >>= 1; r++; }
    for (ll a : bse) {
        ll v = qpow(a, d, p);
        if (v <= 1 || v == p - 1) continue;
        for (int i = 1; i <= r; ++i) {
            v = (__int128)v * v % p;
            if (v == p - 1 && i != r) { v = 1; break; }
            if (v <= 1) return false;
        }
        if (v != 1) return false;
    }
    return true;
}
ll pollard_rho(ll p) {
    if (p == 4) return 2;
    if (is_prime(p)) return p;
    while (1) {
        ll c = 1ll * rand() % (p - 1) + 1;
        auto f = [=](ll x) {
            return ((__int128)x * x + c) % p;
        };
        ll t = 0, r = 0, mul = 1, bd = 1, tmp;
        while (1) {
            for (ll stp = 1; stp <= bd; ++stp) {
                t = f(t);
                mul = (__int128)mul * abs(t - r) % p;
                if (!(stp % 127)) {
                    tmp = __gcd(mul, p);
                    if (tmp > 1) return tmp;
                }
            }
            tmp = __gcd(mul, p);
            if (tmp > 1) return tmp;
            bd <<= 1;
            r = t; mul = 1;
        }
    }
}
void findfac(ll n) {
    if (n == 1 || n <= maxm) return;
    if (is_prime(n)) { maxm = max(n, maxm); return; }
    ll v = pollard_rho(n);
    while (n % v == 0) n /= v;
    findfac(v);
    findfac(n);
}
int main() {
    srand((unsigned)time(NULL));
    ll T, n;
    cin >> T;
    while (T--) {
        cin >> n;
        if (is_prime(n)) printf("Prime\n");
        else {
            maxm = 1;
            findfac(n);
            printf("%lld\n", maxm);
        }
    }
    return 0;
}
\end{minted}


\section{Lucas 定理与 exLucas}
\par \noindent 若 $p \in Prime$,则
$$
\binom n m \equiv \binom{\lfloor n / p \rfloor}{\lfloor m / p \rfloor} \binom{n \bmod p}{m \bmod p} \pmod p
$$
\par \noindent 若 $p \notin Prime$,对 $p$ 进行唯一分解得 $p = \prod {p_i^{c_i}}$,分别求解 $\binom{n}{m}\ mod \ p_i^{c_i}$ 后使用 exCRT 合并,复杂度不超过 $O(\sqrt{P}+\sum p_i^{c_i})\leq O(P)$ 

\par \noindent 输入 $n,m,p$,求解$\ \binom n m \ mod \ p$
\begin{minted}{c++}
#include<bits/stdc++.h>
using namespace std;
typedef long long ll;
typedef __int128 lll;
const ll P = 1e6 + 5;
const ll inf = 2e9;
ll n, m, p, cnt;
// mfac[k][i] denotes i! / p[k]^imax mod p[k]^c[k] , minv[k][i] denotes inverse of mfac[k][i]
ll pc[10], pfac[10], cfac[10], mfac[10][P], minv[10][P];
ll exgcd(ll a, ll b, ll &x, ll &y) {
    if (b == 0) { x = 1; y = 0; return a; }
    ll re = exgcd(b, a % b, x, y), tmp;
    tmp = x; x = y; y = tmp - (a / b) * y;
    return re;
}
ll get_inv(ll a, ll mod) {
    ll x, y, re = exgcd(a, mod, x, y);
    return (x % mod + mod) % mod;
}
ll qpow(ll x, ll exp, ll mod) {
    ll re = 1;
    while (exp) {
        if (exp & 1) re = re * x % mod;
        x = x * x % mod;
        exp >>= 1;
    }
    return re;
}
void init() {
    int ub = sqrt(p);
    for (int i = 2; i <= ub && i <= p; ++i)
        if (p % i == 0) {
            pfac[++cnt] = i;
            while (p % i == 0) {
                p /= i; cfac[cnt]++;
            }
        }
    if (p != 1) pfac[++cnt] = p, cfac[cnt] = 1;
    for (int k = 1; k <= cnt; ++k) {
        pc[k] = qpow(pfac[k], cfac[k], inf);
        mfac[k][0] = minv[k][0] = 1;
        mfac[k][1] = 1;
        for (int i = 2; i <= pc[k]; ++i)
            if (i % pfac[k]) mfac[k][i] = mfac[k][i - 1] * i % pc[k];
            else mfac[k][i] = mfac[k][i - 1];
        minv[k][pc[k]] = get_inv(mfac[k][pc[k]], pc[k]);
        for (int i = pc[k] - 1; i >= 1; --i)
            if ((i + 1) % pfac[k]) minv[k][i] = minv[k][i + 1] * (i + 1) % pc[k];
            else minv[k][i] = minv[k][i + 1];
    }
}
ll exCRT(ll n, ll *a, ll *m) {
    ll ans = a[1], mod = m[1];
    for (ll i = 2; i <= n; ++i) {
        ll x, y, d = exgcd(mod, m[i], x, y), tmp;
        if ((a[i] - ans) % d) return -1;

        lll px = (lll)(a[i] - ans) / d * x;
        tmp = m[i] / d;
        x = (px % tmp + tmp) % tmp;
        ans = ((lll)x * mod + ans) % (mod / d * m[i]);
        mod = mod / d * m[i];
    }
    return ans;
}
ll exlucas() {
    ll ans[10] = {0};
    for (int k = 1; k <= cnt; ++k) {
        ll a = 0, b = 0, c = 0;
        for (ll i = pfac[k]; i <= n; i *= pfac[k]) a += n / i;
        for (ll i = pfac[k]; i <= m; i *= pfac[k]) b += m / i;
        for (ll i = pfac[k]; i <= n - m; i *= pfac[k]) c += (n - m) / i;
        if (a - b - c >= cfac[k]) ans[k] = 0;
        else {
            ans[k] = qpow(pfac[k], a - b - c, pc[k]);
            a = b = c = 0;
            for (ll i = pc[k]; i <= n; i *= pfac[k]) a += n / i;
            for (ll i = pc[k]; i <= m; i *= pfac[k]) b += m / i;
            for (ll i = pc[k]; i <= n - m; i *= pfac[k]) c += (n - m) / i;
            a = qpow(mfac[k][pc[k]], a, pc[k]);
            b = qpow(minv[k][pc[k]], b, pc[k]);
            c = qpow(minv[k][pc[k]], c, pc[k]);
            for (ll i = 1; i <= n; i *= pfac[k])
                a = a * mfac[k][n / i % pc[k]] % pc[k];
            for (ll i = 1; i <= m; i *= pfac[k])
                b = b * minv[k][m / i % pc[k]] % pc[k];
            for (ll i = 1; i <= n - m; i *= pfac[k])
                c = c * minv[k][(n - m) / i % pc[k]] % pc[k];
            ans[k] = ans[k] * (a * b * c % pc[k]) % pc[k]; // care about the overflow
        }
    }
    return exCRT(cnt, ans, pc);
}
int main() {
    cin >> n >> m >> p;
    init();
    printf("%lld\n", exlucas());
    return 0;
}
\end{minted}



\section{二次剩余 - Cipolla}
\begin{tcolorbox}
给出 $N,p$,求解方程 $x^2 \equiv N \pmod{p}$ 保证 $p$ 是奇素数。递增输出两解;若两解相同,只输出其中一个;若无解,则输出 No solution!  。
\end{tcolorbox}
\begin{minted}{c++}
#include <bits/stdc++.h>
using namespace std;
typedef long long ll;
const int MOD = 998244353;
int T;
inline int qpow(int x, int pw, int p = MOD) {
    int res = 1;
    while (pw) {
        if (pw & 1) res = (ll)res * x % p;
        x = (ll)x * x % p;
        pw >>= 1;
    }
    return res;
}
inline int quad_res(int x, int p = MOD) {
    if (x >= MOD) x %= MOD;
    if (x <= 1) return x == 1;
    if (qpow(x, (p - 1) >> 1, p) != 1) return -1;
    // cipolla
    static int square_i, mp;
    int tmp;
    for (int i = 1; i <= p; ++i) {
        square_i = ((ll)i * i + p - x) % p;
        if (qpow(square_i, (p - 1) >> 1, p) == p - 1) {
            tmp = i;
            break;
        }
    }
    mp = p;
    struct M_Complex {
        int a, b;
        M_Complex(int x, int y) {
            this->a = x;
            this->b = y;
        }
        M_Complex operator*(const M_Complex x) {
            int xa = ((ll)this->a * x.a + (ll)square_i * this->b % mp * x.b) % mp;
            int xb = ((ll)this->a * x.b + (ll)(this->b) * x.a) % mp;
            if (xa > mp || xb > mp) {
                printf("huaji\n");
            }
            this->a = xa;
            this->b = xb;
            return *this;
        }
        static M_Complex qpow(M_Complex x, int pw) {
            M_Complex res(1, 0);
            res.a = 1;
            res.b = 0;
            while (pw) {
                if (pw & 1) res = res * x;
                x = x * x;
                pw >>= 1;
            }
            return res;
        }
    };
    M_Complex r(tmp, 1);
    r = M_Complex::qpow(r, (p + 1) >> 1);
    return r.a;
}
int main() {
    cin >> T;
    while (T--) {
        int x, p, tmp;
        cin >> x >> p;
        tmp = quad_res(x, p);
        if (tmp == -1)
            printf("No solution!\n");
        else {
            if (tmp == p - tmp || tmp == 0)
                printf("%d\n", tmp);
            else {
                if (tmp > p - tmp)
                    printf("%d %d\n", p - tmp, tmp);
                else
                    printf("%d %d\n", tmp, p - tmp);
            }
        }
    }
    return 0;
}
\end{minted}


\section{N次剩余}
\begin{tcolorbox}
解方程 $x^n \equiv k(mod m)$ ,其中 $x\in [0, m−1]$ 。给出所有 $c$ 个解
\end{tcolorbox}
\begin{minted}{c++}
#include <bits/stdc++.h>
typedef long long LL;
int A, B, mod;
int pow(int x, int y, int mod = 0, int ans = 1) {
    if (mod) {
        for (; y; y >>= 1, x = (LL)x * x % mod)
            if (y & 1) ans = (LL)ans * x % mod;
    }
    else {
        for (; y; y >>= 1, x = x * x)
            if (y & 1) ans = ans * x;
    }
    return ans;
}
struct factor {
    int prime[20], expo[20], pk[20], tot;
    void factor_integer(int n) {
        tot = 0;
        for (int i = 2; i * i <= n; ++i) if (n % i == 0) {
            prime[tot] = i, expo[tot] = 0, pk[tot] = 1;
            do ++expo[tot], pk[tot] *= i; while ((n /= i) % i == 0);
            ++tot;
        }
        if (n > 1) prime[tot] = n, expo[tot] = 1, pk[tot++] = n;
    }
    int phi(int id) const {
        return pk[id] / prime[id] * (prime[id] - 1);
    }
} mods, _p;

int p_inverse(int x, int id) {
    assert(x % mods.prime[id] != 0);
    return pow(x, mods.phi(id) - 1, mods.pk[id]);
}

void exgcd(int a, int b, int &x, int &y) {
    if (!b) x = 1, y = 0;
    else exgcd(b, a % b, y, x), y -= a / b * x;
}
int inverse(int x, int mod) {
    assert(std::__gcd(x, mod) == 1);
    int ret, tmp;
    exgcd(x, mod, ret, tmp), ret %= mod;
    return ret + (ret >> 31 & mod);
}

std::vector<int> sol[20];

void solve_2(int id, int k) {
    int mod = 1 << k;
    if (k == 0) { sol[id].emplace_back(0); return; }
    else {
        solve_2(id, k - 1); std::vector<int> t;
        for (int s : sol[id]) {
            if (!((pow(s, A) ^ B) & mod - 1))
                t.emplace_back(s);
            if (!((pow(s | 1 << k - 1, A) ^ B) & mod - 1))
                t.emplace_back(s | 1 << k - 1);
        }
        std::swap(sol[id], t);
    }
}

int BSGS(int B, int g, int mod) { // g^x = B (mod M) => g^iL = B*g^j (mod M) : iL - j
    std::unordered_map<int, int> map;
    int L = std::ceil(std::sqrt(mod)), t = 1;
    for (int i = 1; i <= L; ++i) {
        t = (LL)t * g % mod;
        map[(LL)B * t % mod] = i;
    }
    int now = 1;
    for (int i = 1; i <= L; ++i) {
        now = (LL)now * t % mod;
        if (map.count(now)) return i * L - map[now];
    }
    assert(0);
}

int find_primitive_root(int id) {
    int phi = mods.phi(id); _p.factor_integer(phi);
    auto check = [&](int g) {
        for (int i = 0; i < _p.tot; ++i)
            if (pow(g, phi / _p.prime[i], mods.pk[id]) == 1)
                return 0;
        return 1;
    };
    for (int g = 2; g < mods.pk[id]; ++g) if (check(g)) return g;
    assert(0);
}

void division(int id, int a, int b, int mod) { //  ax = b (mod M)
    int M = mod, g = std::__gcd(std::__gcd(a, b), mod);
    a /= g, b /= g, mod /= g;
    if (std::__gcd(a, mod) > 1) return;
    int t = (LL)b * inverse(a, mod) % mod;
    for (; t < M; t += mod) sol[id].emplace_back(t);
}

void solve_p(int id, int B = ::B) {
    int p = mods.prime[id], e = mods.expo[id], mod = mods.pk[id];
    if (B % mod == 0) {
        int q = pow(p, (e + A - 1) / A);
        for (int t = 0; t < mods.pk[id]; t += q)
            sol[id].emplace_back(t);
    }
    else if (B % p != 0) {
        int phi = mods.phi(id);
        int g = find_primitive_root(id), z = BSGS(B, g, mod);
        division(id, A, z, phi);
        for (int &x : sol[id]) x = pow(g, x, mod);
    }
    else {
        int q = 0; while (B % p == 0) B /= p, ++q;
        int pq = pow(p, q);
        if (q % A != 0) return;
        mods.expo[id] -= q, mods.pk[id] /= pq;
        solve_p(id, B);
        mods.expo[id] += q, mods.pk[id] *= pq;
        if (!sol[id].size()) return;

        int s = pow(p, q - q / A);
        int t = pow(p, q / A);
        int u = pow(p, e - q);

        std::vector<int> res;
        for (int y : sol[id]) {
            for (int i = 0; i < s; ++i)
                res.emplace_back((i * u + y) * t);
        }
        std::swap(sol[id], res);
    }
}

std::vector<int> allans;
void dfs(int dep, int ans, int mod) {
    if (dep == mods.tot) { allans.emplace_back(ans); return; }
    int p = mods.pk[dep], k = p_inverse(mod % p, dep);
    for (int a : sol[dep]) {
        int nxt = (LL)(a - ans % p + p) * k % p * mod + ans;
        dfs(dep + 1, nxt, mod * p);
    }
}

void solve() {
    std::cin >> A >> mod >> B, mods.factor_integer(mod);
    allans.clear();
    for (int i = mods.tot - 1; ~i; --i) {
        sol[i].clear();
        mods.prime[i] == 2 ? solve_2(i, mods.expo[i]) : solve_p(i);
        if (!sol[i].size()) { return std::cout << 0 << '\n', void(0); }
    }
    dfs(0, 0, 1), std::sort(allans.begin(), allans.end());
    std::cout << allans.size() << '\n';
    for (int i : allans) std::cout << i << ' '; std::cout << '\n';
}

int main() {
    std::ios::sync_with_stdio(0), std::cin.tie(0);
    int tc; std::cin >> tc; while (tc--) solve();
    return 0;
}
\end{minted}

\section{离散对数同余方程}
\subsection{BSGS(Baby Step Giant Step)}
\par \noindent 求解同余方程 $a^x \equiv b \pmod p$
~\\
\par \noindent 设 $\gcd(a, p) = 1$,令$x = kT - m(T = \lceil \sqrt{p} \rceil, 0 \leq m < T)$。原方程变为 $a^{kT - m} \equiv b \pmod p$
~\\
\par \noindent 由于 $a, p$ 互质,$a$ 存在逆元,于是方程等价于 $(a^T)^{k} \equiv ba^m \pmod p$ 
~\\
\par \noindent 把 $ba^m$ 的所有值插入哈希表,然后枚举 $k$ 即可。复杂度 $O(\sqrt p)$。
\subsection{EXBSGS}

\par \noindent 设 $\gcd(a, p) \neq 1$
~\\
\par \noindent 先验证 $x = 0$ 是否为解,下面假定 $x > 0$。
~\\
\par \noindent 设 $d = \gcd(a, p)$,当 $b \bmod d \neq 0$ 时方程无解。
~\\
\par \noindent 否则方程等价于 $a^{x-1}\dfrac{a}{d} \equiv \dfrac{b}{d} \pmod {\dfrac{p}{d}}$
~\\
\par \noindent 重复此过程,直到底数和模数互质,此时可以用 `BSGS` 求解。
\begin{minted}{c++}
#include<bits/stdc++.h>
typedef long long ll;
using namespace std;
inline int read() {
    char ch = getchar();
    int re = 0;
    while (ch < '0' || ch>'9') ch = getchar();
    while (ch >= '0' && ch <= '9') { re = re * 10 + ch - '0'; ch = getchar(); }
    return re;
}
// struct my_hash {
//     static uint64_t splitmix64(uint64_t x) {
//         x += 0x9e3779b97f4a7c15;
//         x = (x ^ (x >> 30)) * 0xbf58476d1ce4e5b9;
//         x = (x ^ (x >> 27)) * 0x94d049bb133111eb;
//         return x ^ (x >> 31);
//     }
//     size_t operator()(uint64_t x) const {
//         static const uint64_t FIXED_RANDOM =
//             chrono::steady_clock::now().time_since_epoch().count();
//         return splitmix64(x + FIXED_RANDOM);
//     }
//     size_t operator()(pair<uint64_t, uint64_t> x) const {
//         static const uint64_t FIXED_RANDOM =
//             chrono::steady_clock::now().time_since_epoch().count();
//         return splitmix64(x.first + FIXED_RANDOM) ^
//             (splitmix64(x.second + FIXED_RANDOM) >> 1);
//     }
// };
int qpow(int x, int exp, int mod) {
    int re = 1;
    while (exp) {
        if (exp & 1) re = 1ll * re * x % mod;
        x = 1ll * x * x % mod;
        exp >>= 1;
    }
    return re;
}
int exgcd(int a, int b, ll &x, ll &y) {
    if (b == 0) {
        x = 1; y = 0; return a;
    }
    int re = exgcd(b, a % b, x, y);
    ll tmp;
    tmp = x; x = y; y = tmp - (a / b) * y;
    return re;
}
int getinv(int a, int p) {
    ll x, y;
    int re = exgcd(a, p, x, y);
    x = (x % p + p) % p;
    return x;
}
int exBSGS(int a, int b, int p) {
    unordered_map <int, int> mp;
    int d, ofs = 0;
    while (1) {
        a %= p; b %= p;
        d = __gcd(a, p);
        if (b % d) {
            if (b == 1) return ofs;
            else return -1;
        }
        if (d == 1) break;
        p /= d; b = 1ll * b / d * getinv(a / d, p) % p; ofs++;
    }
    int k = sqrt(p), ub = p / k + 1, tmp = 1ll * b * a % p;
    for (int i = 1; i <= k; ++i) {
        mp[tmp] = i;
        tmp = 1ll * tmp * a % p;
    }
    a = qpow(a, k, p); tmp = a;
    for (int i = 1; i <= ub; ++i) {
        if (mp.count(tmp))
            return i * k - mp[tmp] + ofs;
        tmp = 1ll * tmp * a % p;
    }
    return -1;
}
int main() {
    int a, p, b;
    a = read(); p = read(); b = read();
    while (a) {
        int ret = exBSGS(a, b, p);
        if (ret == -1) printf("No Solution\n");
        else printf("%d\n", ret);
        a = read(); p = read(); b = read();
    }
}
\end{minted}