\section{分治NTT(简短)}

\begin{minted}{c++}
#include <bits/stdc++.h>

using namespace std;
const int N = 3e5 + 5, P = 998244353;
using ll = int64_t;

#define inc(a, b) (((a) += (b)) >= P ? (a) -= P : 0)
#define dec(a, b) (((a) -= (b)) < 0 ? (a) += P : 0)
#define mul(a, b) (ll(a) * (b) % P)
int POW(ll a, int b = P - 2, ll x = 1) {
    for (; b; b >>= 1, a = a * a % P)
        if (b & 1) x = x * a % P;
    return x;
}

int inv[N], fac[N], ifac[N], _ = [] {
    fac[0] = fac[1] = ifac[0] = ifac[1] = inv[1] = 1;
    for (ll i = 2; i < N; ++i) {
        fac[i] = (ll)fac[i - 1] * i % P;
        inv[i] = (ll)(P - P / i) * inv[P % i] % P;
        ifac[i] = (ll)ifac[i - 1] * inv[i] % P;
    }
    return 0;
}();

namespace NTT {
const int G = 3, L = 1 << 21;
int W[L], _ = [] {
    W[L / 2] = 1;
    for (int i = L / 2 + 1, wn = POW(G, P / L); i < L; ++i) W[i] = mul(W[i - 1], wn);
    for (int i = L / 2 - 1; ~i; --i) W[i] = W[i << 1];
    return 0;
}();
void dft(int *a, int n) {
    for (int k = n >> 1; k; k >>= 1)
        for (int i = 0; i < n; i += k << 1)
            for (int j = 0; j < k; ++j) {
                int &x = a[i + j], y = a[i + j + k];
                a[i + j + k] = mul(x - y + P, W[k + j]);
                inc(x, y);
            }
}
void idft(int *a, int n) {
    for (int k = 1; k < n; k <<= 1)
        for (int i = 0; i < n; i += k << 1)
            for (int j = 0; j < k; ++j) {
                int x = a[i + j], y = mul(a[i + j + k], W[k + j]);
                a[i + j + k] = x < y ? x - y + P : x - y;
                inc(a[i + j], y);
            }
    for (int i = 0, in = P - (P - 1) / n; i < n; ++i)
        a[i] = mul(a[i], in);
    reverse(a + 1, a + n);
}
} // namespace NTT

int norm(int n) { return 1 << (__lg(n - 1) + 1); }

struct Poly : public vector<int> {
#define T (*this)
    using vector<int>::vector;
    int deg() const { return size(); } // 多项式维度

    Poly &operator^=(const Poly &a) { // 点乘(对应位置相乘)
        if (a.deg() < deg()) resize(a.deg());
        for (int i = 0; i < deg(); ++i) T[i] = mul(T[i], a[i]);
        return T;
    }

    Poly pre(int k) const { return k < deg() ? Poly(begin(), begin() + k) : T; }
    friend void dft(Poly &a) { NTT::dft(a.data(), a.size()); }
    friend void idft(Poly &a) { NTT::idft(a.data(), a.size()); }
    friend Poly conv(Poly a, Poly b, int n) { // 卷积
        a.resize(n), dft(a);
        b.resize(n), dft(b);
        return idft(a ^= b), a;
    }
    Poly operator*(const Poly &a) const { // 多项式 × 多项式 (卷积)
        int n = deg() + a.deg() - 1;
        return conv(T, a, norm(n)).pre(n);
    }
#undef T
};
Poly f, g;
int n;
void solve(int l, int r) {
    if (l == r) return;

    int mid = l + r >> 1;
    solve(l, mid);

    Poly a, b;
    // [l, mid] -> [mid + 1, r]
    for (int i = l; i <= mid; i++) a.push_back(f[i]);
    // 需要 g 的下标范围是 [1, r - l]
    for (int i = 1; i <= r - l; i++) b.push_back(g[i]);

    a = a * b;
    // a[0] 是 f[l] 和 g[1] 卷积出来 原偏移量需要 - (l + 1)
    for (int i = mid + 1; i <= r; i++) f[i] = (1ll * f[i] + a[i - l - 1]) % P;
    solve(mid + 1, r);
}
int main() {
    cin >> n;
    f.resize(n), g.resize(n);
    f[0] = g[0] = 1;

    for (int i = 1; i < n; i++) cin >> g[i];

    solve(0, n - 1);

    for (int i = 0; i < n; i++) cout << f[i] << ' ';
    return 0;
}
\end{minted}

\section{多项式桶 - zyn1.0}

\begin{minted}{c++}
#include <bits/stdc++.h>
using namespace std;
typedef vector<int> poly;
typedef long long ll;

constexpr int N = 262144 + 5, G = 3, invG = 332748118, MOD = 998244353;
int w[2][N << 1], rev[N], inv[N];
inline int qpow(int x, int pw, int p = MOD) {
    int res = 1;
    while (pw) {
        if (pw & 1) res = (ll)res * x % p;
        x = (ll)x * x % p;
        pw >>= 1;
    }
    return res;
}
// Cipolla 解二次剩余
inline int quad_res(int x, int p = MOD) {
    if (x >= p) x %= p;
    if (x == 0 || x == 1) return x;
    if (qpow(x, (p - 1) >> 1, p) != 1) return -1;
    static int i_squared, mp = p;
    int tmp;
    for (int i = 1; i <= p; ++i) {
        i_squared = ((ll)i * i + p - x) % p;
        if (qpow(i_squared, (p - 1) >> 1, p) == p - 1) {
            tmp = i;
            break;
        }
    }
    struct M_Complex {
        int a, b;
        M_Complex(int x, int y) {
            this->a = x;
            this->b = y;
        }
        M_Complex operator*(const M_Complex x) {
            int xa = ((ll)this->a * x.a + (ll)i_squared * this->b % mp * x.b) % mp;
            int xb = ((ll)this->a * x.b + (ll)this->b * x.a) % mp;
            this->a = xa;
            this->b = xb;
            return *this;
        }
        static M_Complex qpow(M_Complex x, int pw) {
            M_Complex res(1, 0);
            while (pw) {
                if (pw & 1) res = res * x;
                x = x * x;
                pw >>= 1;
            }
            return res;
        }
    };
    M_Complex r(tmp, 1);
    r = M_Complex::qpow(r, (p + 1) >> 1);
    return min(r.a, p - r.a);
}
void poly_init(int upb = 200005) {
    int lim = 1;
    while (lim <= upb) lim <<= 1;
    for (int k = 1; k <= lim; k <<= 1) {
        int wk = qpow(G, (MOD - 1) / k), iwk = qpow(wk, MOD - 2);
        w[0][k] = w[1][k] = 1;
        for (int i = 1; i < k; ++i) {
            w[0][k + i] = (ll)w[0][k + i - 1] * wk % MOD;
            w[1][k + i] = (ll)w[1][k + i - 1] * iwk % MOD;
        }
    }
    inv[0] = inv[1] = 1;
    for (int i = 2; i < lim; ++i) inv[i] = (ll)(MOD - MOD / i) * inv[MOD % i] % MOD;
}
void poly_print(const poly &a) {
    int la = a.size();
    for (int i = 0; i < la; ++i) printf("%d ", a[i]);
    printf("\n");
}
void NTT(poly &a, int lim, int opt) {
    for (int i = 0; i < lim; ++i)
        if (i < rev[i]) swap(a[i], a[rev[i]]);
    for (int k = 2; k <= lim; k <<= 1)
        for (int i = 0; i < lim; i += k)
            for (int j = 0; j < (k >> 1); ++j) {
                int u = a[i + j], v = (ll)w[opt][k + j] * a[i + j + (k >> 1)] % MOD;
                a[i + j] = (u + v) % MOD;
                a[i + j + (k >> 1)] = (u - v + MOD) % MOD;
            }
    if (opt) {
        int invl = qpow(lim, MOD - 2);
        for (int i = 0; i < lim; ++i) a[i] = (ll)a[i] * invl % MOD;
    }
}
// 以下方法视需求使用
poly operator+(const poly &a, const int &b) {
    poly res = a;
    if (!res.size())
        res.push_back(b);
    else
        res[0] = (res[0] + b) % MOD;
    return res;
}
poly operator+(const int &a, const poly &b) {
    return b + a;
}
poly operator+(const poly &a, const poly &b) {
    int lb = b.size();
    poly res = a;
    if (res.size() < lb) res.resize(lb);
    for (int i = 0; i < lb; ++i) res[i] = (res[i] + b[i]) % MOD;
    return res;
}
poly operator-(const poly &a, const int &b) {
    return a + (MOD - b);
}
poly operator-(const int &a, const poly &b) {
    int lb = b.size();
    poly res(lb);
    for (int i = 0; i < lb; ++i) res[i] = MOD - b[i];
    res[0] = (a + res[0]) % MOD;
    return res;
}
poly operator-(const poly &a, const poly &b) {
    int lb = b.size();
    poly res = a;
    if (res.size() < lb) res.resize(lb);
    for (int i = 0; i < lb; ++i) res[i] = (res[i] + MOD - b[i]) % MOD;
    return res;
}
poly operator*(const poly &a, const int &b) {
    int la = a.size();
    poly res(la);
    for (int i = 0; i < la; ++i) res[i] = (ll)a[i] * b % MOD;
    return res;
}
poly operator*(const int &a, const poly &b) {
    return b * a;
}
// 多项式乘法
poly poly_mul(const poly &a, const poly &b, int deg = -1) {
    poly f = a, g = b;
    if (deg == -1) deg = f.size() + g.size() - 2;
    if (f.size() > deg + 1) f.resize(deg + 1);
    if (g.size() > deg + 1) g.resize(deg + 1);
    int lim = 1, len = 0, upb = f.size() + g.size() - 2;
    while (lim <= upb) {
        lim <<= 1;
        len++;
    }
    f.resize(lim);
    g.resize(lim);
    for (int i = 1; i < lim; ++i)
        rev[i] = (rev[i >> 1] >> 1) | ((i & 1) << (len - 1));
    NTT(f, lim, 0);
    NTT(g, lim, 0);
    for (int i = 0; i < lim; ++i) f[i] = (ll)f[i] * g[i] % MOD;
    NTT(f, lim, 1);
    f.resize(deg + 1);
    return f;
}
// 多项式逆元 (mod x^(deg+1)),deg >= 0
poly poly_inv(const poly &a, int deg = -1) {
    if (deg == -1) deg = a.size() - 1;
    poly f, res(1, qpow(a[0], MOD - 2));
    int now = 0, lim = 2, len = 1;
    while (now < deg) {
        now = (now << 1) + 1;
        lim <<= 1;
        len++;
        if (now > a.size() - 1)
            f.assign(a.begin(), a.end());
        else
            f.assign(a.begin(), a.begin() + now + 1);
        f.resize(lim);
        res.resize(lim);
        for (int i = 1; i < lim; ++i)
            rev[i] = (rev[i >> 1] >> 1) | ((i & 1) << (len - 1));
        NTT(f, lim, 0);
        NTT(res, lim, 0);
        for (int i = 0; i < lim; ++i) res[i] = (ll)res[i] * (MOD + 2 - (ll)f[i] * res[i] % MOD) % MOD;
        NTT(res, lim, 1);
        res.resize(now + 1);
    }
    res.resize(deg + 1);
    return res;
}
// 多项式除法
poly poly_div(const poly &a, const poly &b) {
    int rdeg = a.size() - b.size();
    poly res;
    if (rdeg < 0) perror("Wrong div");
    poly f = a, g = b;
    reverse(f.begin(), f.end());
    reverse(g.begin(), g.end());
    res = poly_mul(f, poly_inv(g, rdeg), rdeg);
    reverse(res.begin(), res.end());
    return res;
}
// 多项式开方,需满足a[0]为二次剩余
poly poly_sqrt(const poly &a, int deg = -1) {
    if (deg == -1) deg = a.size() - 1;
    poly f, res(1, quad_res(a[0], MOD));
    int now = 0, lim = 2, len = 1;
    while (now < deg) {
        now = (now << 1) + 1;
        lim <<= 1;
        len++;
        if (now > a.size() - 1)
            f.assign(a.begin(), a.end());
        else
            f.assign(a.begin(), a.begin() + now + 1);
        res = inv[2] * (poly_mul(f, poly_inv(res, now), now) + res);
        res.resize(now + 1);
    }
    res.resize(deg + 1);
    return res;
}
// 多项式求导
poly poly_derive(const poly &a) {
    int la = a.size();
    poly res;
    if (la == 1) {
        res.push_back(0);
    }
    else {
        res.resize(la - 1);
        for (int i = 1; i < la; ++i) res[i - 1] = (ll)a[i] * i % MOD;
    }
    return res;
}
// 多项式积分
poly poly_integrate(const poly &a) {
    int la = a.size();
    poly res(la + 1);
    for (int i = 0; i < la; ++i) res[i + 1] = (ll)a[i] * inv[i + 1] % MOD;
    return res;
}
// 多项式对数
poly poly_ln(const poly &a, int deg = -1) {
    if (deg == -1) deg = a.size() - 1;
    return poly_integrate(poly_mul(poly_derive(a), poly_inv(a, deg - 1), deg - 1));
}
// 多项式指数
poly poly_exp(const poly &a, int deg = -1) {
    if (deg == -1) deg = a.size() - 1;
    poly f, res(1, 1);
    int now = 0, lim = 2, len = 1;
    while (now < deg) {
        now = (now << 1) + 1;
        if (now > a.size() - 1)
            f.assign(a.begin(), a.end());
        else
            f.assign(a.begin(), a.begin() + now + 1);
        res = poly_mul(res, f - poly_ln(res, now) + 1, now);
    }
    res.resize(deg + 1);
    return res;
}
// 多项式快速幂 - pw 为 const char*
poly poly_qpow(const poly &a, char *cpw, int deg = -1) {
    if (deg == -1) deg = a.size() - 1;
    poly res;
    if (strlen(cpw) == 1 && cpw[0] == '0') {
        res.push_back(1);
        res.resize(deg + 1);
    }
    else {
        poly f = a;
        int lf = f.size(), t = lf, lcpw = strlen(cpw);
        ll ppw = 0, kpw = 0;
        bool zflag = false;
        for (int i = 0; i < lf; ++i) {
            if (f[i]) {
                t = i;
                break;
            }
        }
        if (t == lf)
            zflag = true;
        else {
            for (int i = 0; i < lcpw; ++i) {
                ppw = ppw * 10 + cpw[i] - '0';
                kpw = kpw * 10 + cpw[i] - '0';
                if (ppw * t > deg) {
                    zflag = true;
                    break;
                }
                if (ppw >= MOD) ppw %= MOD;
                if (kpw >= MOD - 1) kpw %= MOD - 1;
            }
        }
        if (zflag)
            res.resize(deg + 1);
        else {
            f.erase(f.begin(), f.begin() + t);
            int k = f[0], zbound = ppw * t;
            f = f * qpow(k, MOD - 2);
            f = poly_exp((int)ppw * poly_ln(f, deg - zbound), deg - zbound) * qpow(k, kpw);
            res.resize(zbound);
            res.insert(res.end(), f.begin(), f.end());
        }
    }
    return res;
}
// 多项式三角函数
poly poly_sin(const poly &a, int deg = -1) {
    if (deg == -1) deg = a.size() - 1;
    int i = qpow(G, (MOD - 1) >> 2);
    poly res = poly_exp(i * a, deg);
    return qpow(2 * i, MOD - 2) * (res - poly_inv(res, deg));
}
poly poly_cos(const poly &a, int deg = -1) {
    if (deg == -1) deg = a.size() - 1;
    int i = qpow(G, (MOD - 1) >> 2);
    poly res = poly_exp(i * a, deg);
    return inv[2] * (res + poly_inv(res, deg));
}
// 多项式反三角函数
poly poly_arcsin(const poly &a, int deg = -1) {
    if (deg == -1) deg = a.size() - 1;
    return poly_integrate(poly_mul(poly_derive(a), poly_inv(poly_sqrt(1 - poly_mul(a, a, deg), deg), deg), deg - 1));
}
poly poly_arctan(const poly &a, int deg = -1) {
    if (deg == -1) deg = a.size() - 1;
    return poly_integrate(poly_mul(poly_derive(a), poly_inv(1 + poly_mul(a, a, deg), deg), deg - 1));
}
inline int read() {
    char ch = getchar();
    int re = 0;
    while (ch < '0' || ch > '9') ch = getchar();
    while (ch >= '0' && ch <= '9') {
        re = re * 10 + ch - '0';
        ch = getchar();
    }
    return re;
}
// 多项式多点求值 - 转置原理
int tr[N << 1];
void build() {

}
int main() {
    //freopen("1.in", "r", stdin);
    int n, k;
    char ck[15];
    poly a;
    poly_init();
    n = read();
    k = read();
    sprintf(ck, "%d", k);
    for (int i = 0; i <= n; ++i) a.push_back(read());
    poly r = poly_derive(poly_qpow(poly_ln(a + 2 - a[0] - poly_exp(poly_integrate(poly_inv(poly_sqrt(a))))) + 1, ck));
    r.resize(n);
    poly_print(r);
    return 0;
}
\end{minted}