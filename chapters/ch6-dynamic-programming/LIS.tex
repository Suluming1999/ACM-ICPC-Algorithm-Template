\section{最长上升子序列(LIS)}
\paragraph{朴素做法} $O(n^2)$:$f(i)$ 表示以 $a_i$ 结尾的 LIS 长度,则有状态转移方程 $f(i) = max\{f(j)\} + 1, 1 \le j < i$.
\paragraph{优化做法}  $O(nlogn)$
\begin{itemize}
    \item 设置一个单调栈(满足栈底到栈顶的元素单调递增)s,然后将第一个元素加入栈中。
    \item 接下来开始逐个加入数列中的元素,设当前待入栈的元素为 ai.若 ai > 栈顶元素 s[top], 则直接让 ai 入栈.
    \item 若 ai <= 栈顶元素 s[top], 则在栈中二分查找到第一个小于等于 ai 的元素的位置 pos,将 s[pos] 替换为 ai.
    \item 重复上述步骤,直至所有数都被处理完成.
    \item 此时栈中的元素个数 s.size 即为 LIS 的答案,但注意栈中元素并不是组成 LIS 的元素。
\end{itemize}


\begin{minted}{c++}
// 写法 1,不需要输出方案
int stk[MAXN], top = 0;
vector<int> a;
stk[top = 1] = a[0];
for (int i = 1; i < a.size(); i++)
    if (a[i] > stk[top]) // 严格上升
        stk[++top] = a[i];
    else {
        int pos = lower_bound(stk + 1, stk + top + 1, a[i]) - stk;
        stk[pos] = a[i];
    }
int ans = top;        // stk.size 即为答案
\end{minted}
\paragraph{输出方案} 单调栈 s 数组保存最长上升子序列的长度,设置一个 $pos$ 数组,记录一下数组 $a$ 中每个元素在 $s$ 数组中出现的位置;然后从数组 $a$ 的最后一个元素开始到第一个元素寻找最长上升子序列。
\begin{minted}{c++}
// 写法 2,可以输出方案
int pos[maxn], ans[maxn];
void lis(int a[], int stk [], int n){
    stk[1] = a[1], pos[1] = 1;
    int top = 1;
    for(int i = 2; i <= n; i++) {
        if(a[i] > stk[top])
            stk[++top] = a[i], pos[i] = top;
        else {
            int p = lower_bound(stk + 1, stk + top + 1, a[i]) - stk;
            stk[p] = a[i], pos[i] = p;   // 记录原数组中每个元素在 stk 数组中出现的位置
        }
    }
    int maxx = n + 1; // INT_MAX
    for(int i = n; i >= 1; i--) {
        if (top == 0)
            break;
        if(pos[i] == top && maxx > a[i])
            ans[top] = i, top--, maxx = a[i];
    }
}
\end{minted}
