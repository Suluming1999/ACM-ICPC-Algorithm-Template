\section{数位 DP}
\paragraph{问题场景} 处理出某一区间 $[l, r]$ 范围内,满足条件的数的个数。

\paragraph{一般解法}

\begin{itemize}
    \item \textbf{数位处理}:处理不多于 $i$ 位的数中,有多少个数满足条件,用 $dp[i]$ 表示。
    \item \textbf{状态拓展}:大部分题目与数字本身有一定的关系(不能出现/必须出现特定数字),则状态数组需要多加一维/多维进行转移:$dp[i][j]$ 表示第 $i$ 位数字为 $j$ 且满足条件的数字个数。
    \item \textbf{处理区间端点}:用预处理的信息计算出 $0 \sim r$ 和 $0 \sim l-1$ 闭区间满足条件的数字个数,然后求解 $[l, r]$ 范围的答案,即为 $r$ 端的答案减去 $l-1$ 端的答案。
    \item \textbf{具体实现}:首先在第当前考虑的第 $i$ 位上固定一个数字,那么后面就可以随便填。
\end{itemize}

\paragraph{DFS 函数的参量}
\begin{itemize}
\item 基本量:数字位数 $pos$,最高位限制 $lim$
\item 判断前导 0 的标志:$lead$
\item 一般需要记录数位中的前一位(或前几位):$pre$
\item 其它用于区分状态的参量
\end{itemize}

\paragraph{最高位标记}

当给定的区间 $[0, r]$ 不同时,数位 DP 统计时的位数限制也不同,如 $r=1234$ 那么,当第一位为 1 的时候,第二位的取值范围 $[0, 2]$;当第一位为 0 的时候第二位则可以取 $[0, 9]$。为了分清这样的情况,引入 $lim$ 变量:
\begin{itemize}
    \item 若当前位 $lim = 1$ ,且已经取到了当前位可以取得的最大数字,则下一位 $lim = 1$
    \item 若当前位 $lim = 1$,但取到的数字小于可以取得的最大数字,则下一位 $lim = 0$
    \item 若当前位 $lim = 0$,则下一位 $lim = 0$
\end{itemize}

\paragraph{记忆化搜索}DFS 的时候,可以将已经搜索过的状态记录下来,那么下一次在遇到\textbf{完全相同}的状态的时候,就可以直接返回;所以 DP 数组的维度需要和 DFS 函数的参量(除了 $lim, lead$)相同。

\par \textbf{DFS 参量完全相同} 是状态相同的必要非充分条件。

\begin{minted}{c++}
typedef long long ll;
ll dp[MAXL][MAXD][2][2];

/**
* pos: 当前枚举的位
* pre: 之前的状态,例如上一位,视需要的状态不同可能有不同的保存方式
* lead: 是否有前导 0
* limit: 当前位是否有限制
*/
ll dfs(int pos, int pre, bool lead, bool limit) {
    // 递归边界,返回 1 表示枚举得当前数合法
    if (pos == 0)
        return 1;
    // 满足记忆化条件则可以返回
    if (dp[pos][pre] != -1)
        return dp[pos][pre][limit][lead];
    ll cur = 0;
    // 枚举当前位
    int maxd = limit ? a[pos] : 9;
    for (int i = 0; i <= maxd; i++) {
        // 需要针对多种情况具体分析,例如是否有限制,是否有前导 0 的影响等等
        if (lead && !i)
            cur += dfs(pos - 1, i, true, i == maxd && limit);

        else if (....)
            cur += ...
    }
    // 记忆化存下结果
    return dp[pos][pre][limit][lead] = cur;
}

ll solve(ll x) {
    int pos = 0, init_state = ?;        // init_state 表示初始位之前的状态
    // 拆数
    memset(dp, -1, sizeof dp);
    while (x) {
        a[++pos] = x % 10, \
        x /= 10;
    }
    return dfs(pos, init_state, true, true);
}
\end{minted}

\par \noindent 一般的数位dp的进制很小,比如常见的10进制在dfs会枚举 $[0,9]$,但如果进制数是B,如果进制数非常大,那么在枚举 $[0, B)$ 时复杂度很高,一个优化时不难发现往往只有某些为会影响到lead和limit一些条件,其余的都是相同的lead和limit,可以同时计算。
\begin{tcolorbox}
\par \noindent 求 $\sum_{i=l}^r f^k(i, b, d)$ 表示用 $\mathrm{b}$ 进制表示i,数位数位的次数 数据范围: $1 \leq b \leq 10^9, 0 \leq d<b, 0 \leq k \leq 10^9, 1 \leq l \leq r \leq 10^{18}$ 并且规定 $0^0=0$
\end{tcolorbox}

\begin{minted}{c++}
int dfs(int pos, int s, bool limit, bool lead) {
    if (!pos) return pw[s];
    if (f[pos][s][limit][lead] != -1) return f[pos][s][limit][lead];

    int &res = f[pos][s][limit][lead];
    res = 0;
    vector<int> v; // 存一些可能会改变limit 和 lead 值的"关键位"
    v.push_back(-1);
    v.push_back(0);
    v.push_back(a[pos] - 1);
    v.push_back(a[pos]);
    v.push_back(d - 1);
    v.push_back(d);
    v.push_back(B - 1);
    sort(v.begin(), v.end());
    v.resize(unique(v.begin(), v.end()) - v.begin());

    for (int i = 1; i < (int)v.size(); i++) {
        if (limit && v[i] > a[pos]) break;
        if (lead)
            res = (res + 1ll * (v[i] - v[i - 1]) \ // 一类的同时计算 用做差的形式统计
                * dfs(pos - 1, s + (d == v[i]) && (v[i] != 0)\
                , limit && (v[i] == a[pos]), v[i] == 0) % mod) % mod;
        else
            res = (res + 1ll * (v[i] - v[i - 1]) \
                * dfs(pos - 1, s + (d == v[i])\
                , limit && (v[i] == a[pos]), 0) % mod) % mod;
    }

    return res;
}

int solve(int x) {
    memset(f, -1, sizeof f);
    int cnt = 0;

    while (x) a[++cnt] = x % B, x /= B;

    for (int i = 1; i <= cnt; i++) pw[i] = POW(i, k); // 快速幂预处理

    return dfs(cnt, 0, 1, 1);
}
\end{minted}
