\section{对拍}

\subsection{生成随机数据}

\begin{minted}{c++}
#include <iostream>
#include <cstdlib>
#include <ctime>
using namespace std;
int random(int n) { //生成一个[0,n-1]范围内的数
    return (long long)rand() * rand() % n;
}
int main() {
    srand((unsigned)time(0));
    // ===========随机生成排列===========
    int n = random(100000) + 1, a[100050];

    for (int i = 1; i <= n; i++)
        a[i] = i;

    random_shuffle(a + 1, a + n + 1); //库文件algorithm

    for (int i = 1; i <= n; i++)
        printf("%d ", a[i]);

    //===========随机生成m个[1,n]的子区间===========
    for (int i = 1; i <= m; i++) {
        int l = random(n) + 1;
        int r = random(n) + 1;

        if (l > r)
            swap(l, r);

        printf("%d %d\n", l, r);
    }

    //===========随机生成一棵n个点带边权(<=100000)的树===========
    for (int i = 2; i <= n; i++) {
        int fa = random(i - 1) + 1;
        int val = random(100000) + 1;
        printf("%d %d %d\n", fa, i, val);
    }

    //随机生成一张n个点,m条边的无向图.
    pair<int, int> e[]; //[]内填写数组大小
    map<pair<int, int>, bool> h; //库文件map
    //先生成一棵树,保证联通

    int n = random(具体大小), m = random(具体大小);
    printf("%d %d\n", n, m);

    for (int i = 1; i < n; i++) {
        int fa = random(i) + 1;
        e[i] = make_pair(fa, i + 1);
        h[e[i]] = h[make_pair(i + 1, fa)] = 1;
    }

    //在生成剩余的m-n+1条边
    for (int i = n; i <= n; i++) {
        int x, y;

        do {
            x = random(n) + 1, y = random(n) + 1;
        } while (x == y || h[make_pair(x, y)]);

        e[i] = make_pair(x, y);
        h[e[i]] = h[make_pair(y, x)] = 1;
    }

    //随机打乱,输出
    random_shuffle(e + 1, e + m + 1); //库文件algorithm

    for (int i = 1; i <= m; i++)
        printf("%d %d\n", e[i].first, e[i].second);

    //===========生成一条有n个节点的链===========
    int n = random(1000) + 1;
    printf("%d\n", &n);
    int root = random(n) + 1;
    bool vis[100000];
    vis[root] = 1;
    int last = root;

    for (int i = 1; i < n; i++) {
        int x = random(n) + 1;

        while (vis[x] == 1)
            x = random(n) + 1;

        printf("%d %d\n", last, x);
        last = x;
        vis[x] = 1;
    }

    //===========生成一条有n个节点的菊花图===========
    int n = random(1000) + 1;
    printf("%d\n", n);
    int root = random(n) + 1;

    for (int i = 1; i <= n; i++) {
        if (i == root)
            continue;

        printf("%d %d\n", root, i);
    }

    return 0;
}
\end{minted}

\subsection{Windows下的批处理}

\par \noindent 用文本编辑器(记事本就行)写好,保存为.bat 后缀名
\begin{minted}{c++}
@echo off                               //关掉输入显示,否则所有的命令也会显示出来
:loop                                   //生成随机输入
    rand.exe > in.txt
    my.exe < in.txt  > out.txt       
    std.exe < in.txt  > stdout.txt
    fc out.txt stdout.txt               //比较文件
    if not errorlevel 1   goto loop     //不为1继续循环,fc在文件相同时返回0,不同时返回1
pause                                   //不同时暂停,你可以看in.txt里的数据
goto loop                               //看完数据,按任意键结束暂停,继续循环
\end{minted}

\subsection{Linux下的Bash脚本}

\par \noindent 同样用文本编辑器写好保存为.sh(例如cmp.sh),在执行chmod +x cmp.sh,即可用./cmp.sh来执行它,当然扩展名也不是必需的,完全可以用不带扩展名的cmp命名。
\begin{minted}{c++}
#!/bin/bash
while true; do
    ./r > input                         //生成随机事件
    ./a < input > output.a
    ./b < input > output.b
    diff output.a output.b           //文本比较
    if [ $? -ne 0 ] ; then break;fi     //判断返回值
done
\end{minted}


