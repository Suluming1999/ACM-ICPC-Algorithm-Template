\section{二分图}
\subsection{匈牙利算法}
\par \noindent 考虑点集A,B二分图
\par \noindent \textbf{最小点覆盖}:
\par \noindent 概念:用最少的点覆盖二分图中所有边。
\par \noindent 结论:最小覆盖点=最大匹配
\par \noindent 
\par \noindent \textbf{最小边覆盖}:
\par \noindent 概念:用最少的边覆盖点集A,B中的所有点。
\par \noindent 结论:最小边覆盖=总点数-最大匹配
\par \noindent 
\par \noindent \textbf{最大独立集}
\par \noindent 概念:选出最多的点使得点集内部没有边。
\par \noindent 结论:最大独立集=总点数-最大匹配
\par \noindent 
\par \noindent \textbf{最小路径点覆盖}
\par \noindent 概念:对于一个有向图,选出最少的不相交路径使其覆盖所有点。
\par \noindent 结论:最小路径覆盖=总点数-最大匹配
\par \noindent 
\par \noindent \textbf{最小路径重复点覆盖}
\par \noindent 概念:对于一个有向图,选出最少的(能相交)路径使其覆盖所有点。
\par \noindent 结论:对原图求传递闭包之后再对新图求最小不相交路径覆盖

\begin{minted}{c++}
int match[manx], vis[manx];
bool dfs(int x) {
// 遍历 x 的每条出边
    for (int i = h[x], y; i != -1; i = ne[i]) {
        y = e[i];
        // 如果在当前递归 DFS 的过程中,y 没有被访问过
        if (!vis[y]) {
            vis[y] = 1; // 先将 y 分配给 x, 标记 y 被访问
        // 如果 y 没有被匹配,那就让它与 x 匹配
        // 否则,尝试对 y 已匹配的边匹配其他的节点,然后再让 y 与 x 匹配
        if (!match[y] || dfs(match[y])) {
                match[y] = x;
                return true;
            }
        }
    }
    // 如果走到了这里,说明该点匹配失败
    return false;
}
int main() {
    int ans = 0;
    for (int i = 1; i <= n; i++) {
    // 重设 vis 数组,表示所有点都不在增广路中
        memset(vis, 0, sizeof vis);
    // 尝试为每个点匹配(找增广路),如果匹配成功则 ans++
    if (dfs(i)) ans++;
    }
}
\end{minted}

\subsection{KM算法}
\par \noindent 解决二分图最大带权匹配,时间复杂度 $O(n^3)$.
\begin{minted}{c++}
#include<cstdio>
#include<cstring>
#include<iostream>
#include<algorithm>
using namespace std;
typedef long long ll;
const int N=510;
ll w[N][N],lx[N],ly[N];//顶标
bool visx[N],visy[N];//记录是否在交错树上
int match[N];//匹配
int n,m,p[N];
ll delta,c[N];//delta和更新后的delta
void bfs(int x)
{
    int a,y=0,y1=0;
    for(int i=1;i<=n;i++) p[i]=0,c[i]=1e18;
    match[y]=x;
    do{
        a=match[y],delta=1e18,visy[y]=1;
        for(int b=1;b<=n;b++)
            if(!visy[b])
            {
                if(c[b]>lx[a]+ly[b]-w[a][b])
                    c[b]=lx[a]+ly[b]-w[a][b],p[b]=y;
                if(c[b]<delta)//Δ还是取最小的
                    delta=c[b],y1=b;
            }
        for(int b=0;b<=n;b++)
            if(visy[b])
                lx[match[b]]-=delta,ly[b]+=delta;
            else
                c[b]-=delta;
        y=y1;
    }while(match[y]);
    while(y) match[y]=match[p[y]],y=p[y];
}
ll KM()
{
    for(int i=1;i<=n;i++)
        match[i]=lx[i]=ly[i]=0;
    for(int i=1;i<=n;i++)
    {
        memset(visy,0,sizeof visy);
        bfs(i);
    }
    
    ll res=0;
    for(int i=1;i<=n;i++)
        res+=w[match[i]][i];
    return res;
}
int main()
{
    scanf("%d%d",&n,&m);
    for(int i=1;i<=n;i++)
        for(int j=1;j<=n;j++) 
            w[i][j]=-1e18;
    for(int i=1;i<=m;i++)
    {
        int a,b;ll c;
        scanf("%d%d%lld",&a,&b,&c);
        w[a][b]=max(w[a][b],c);
    }
    printf("%lld\n",KM());
    for(int i=1;i<=n;i++)
        printf("%d ",match[i]);
    puts("");
    return 0;
}
\end{minted}

\subsection{染色判二分图}
\begin{minted}{c++}
int vis[MAXN], flag = 1;
void dfs(int x, int co) {
    vis[x] = co;
    for (int i = h[x]; i != -1; i = ne[i]) {
        if (vis[e[i]] == 0)
            dfs(vis[e[i]], 3 - co); // 注意这里的小技巧,3-1=2, 3-2=1.
        else if (vis[e[i]] == co) {
            flag = 0;
            return;
        }
    }
}
for (int i = 0; i < n; i++) {
    if (!vis[i] && flag)
        dfs(i, 1);
    if (!flag) break;
}
printf(flag ? "Yes" : "No");
\end{minted}
\clearpage