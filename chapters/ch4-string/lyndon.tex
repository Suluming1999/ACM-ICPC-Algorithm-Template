\section{Lyndon 分解}

\paragraph{Lyndon 串的定义} 对于字符串 $s$,如果 $s$ 的字典序严格小于 $s$ 的所有后缀的字典序,我们称 $s$ 是简单串,或者 Lyndon 串 。

\paragraph{Lyndon 分解} $s$ 的 Lyndon 分解记为 $s = w_1w_2...w_k$,其中所有的 $w_i$ 为简单串,且字典序非单调递增,即 $w_{i-1} \ge w_i$。

\begin{minted}{c++}
namespace Lyndon {
    vector<string> duval(string s) {
        int n = s.size(), i = 0;
        vector<string> res;
        while (i < n) {
            int j = i + 1, k = i;
            while (j < n && s[k] <= s[j]) {
                if (s[k] < s[j])
                    k = i;
                else
                    k++;
                j++;
            }
            while (i <= k)
                res.emplace_back(s.substr(i, j - k)), i += j - k;
        }
        return res;
    }
    // 使用 Lyndon 分解求最小表示
    string minimum_representation(string s) {
        s += s;
        int n = s.size();
        int i = 0, ans = 0;
        while (i < n / 2) {
            ans = i;
            int j = i + 1, k = i;
            while (j < n && s[k] <= s[j]) {
                if (s[k] < s[j])
                    k = i;
                else
                    k++;
                j++;
            }
            while (i <= k) i += j - k;
        }
        return s.substr(ans, n / 2);
    }    
}
\end{minted}