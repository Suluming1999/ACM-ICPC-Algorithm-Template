\section{AC 自动机}

\par 在 Trie 树的基础上,为节点增加 fail 指针;当前节点失配的时候,将匹配指针转移到 fail 指针指向的节点。
\vspace{0.5em}

\par\textbf{建树}
\begin{itemize}
    \item 根节点指向的所有节点的 $fail$ 指针都指向根节点
    \item 不存在的节点,$fail$ 指针指向根节点
    \item 普通节点,字符为 $s$ 的 $fail$ 指针,指向它的父节点的 $fail$ 指针指向节点 $fail[p]$ 沿 $s$ 走到的节点。
\end{itemize}

\par\textbf{匹配}
\begin{itemize}
    \item 如果走到了不存在的节点,则将匹配指针移到 fail 指针指向的节点
    \item 从根节点开始匹配,原理与 Trie 树相同,匹配指针沿着 $p[i]$ 所在的字母向下走
    \item 如果失配,则沿着 fail 指针移动,若匹配上则继续匹配,否则不断沿着 fail 指针走。
\end{itemize}

\par\textbf{Fail 指针}
\begin{itemize}
    \item 每个节点 $s$ 有一个失配指针 $p$ ,所有的 $s$ 和它们的 $p$ 构成的树形结构称为 fail 树。
    \item fail 树上每个节点所代表的字符串,是其所有子树所代表的字符串的后缀 ⇒ 一个节点所有祖先,代表的字符串都是这个节点代表的字符串的后缀,如下图所示。
    \item 重要性质:每个节点的 fail 指针,都指向当前节点代表的字符串的最长后缀(如果存在)。
\end{itemize}

\par\textbf{Fail 指针的用法}
\begin{itemize}
    \item 统计每个模式串 p 在文本串 t 当中出现的次数:将 t 在 AC 自动机的上匹配同时建立 fail 树,当经过某个节点时,对答案的贡献为:这个节点所有祖先的权值之和。利用树上差分将经过的所有节点计数 + 1。
    \item 一个模式串 $p_i$ 在其它模式串中出现的次数统计:如果 $p_i$ 在其它的模式串中出现,那么其它模式串的链上一定有一个节点的 fail 指针指向该节点,直接统计该节点在 fail 树上的子节点个数即可。
\end{itemize}



\begin{minted}{c++}
namespace ACAM { 
    struct Trie {
        int ch[26],fail,cnt;
    }trans[N];
    int ins(char *s) {
        int p=0;
        for(int i=0;s[i];i++) {
            int c=s[i]-'a';
            if(!trans[p].ch[c]) trans[p].ch[c]=++cnt;
            p=trans[p].ch[c];
        }
        return p;
    }
    void getfail() {
        queue<int> q;
        for(int i=0;i<26;i++)
            if(trans[0].ch[i]) q.push(trans[0].ch[i]);
            
        while(q.size()) {
            int u=q.front();q.pop();
            for(int i=0;i<26;i++) {
                int &ch=trans[u].ch[i];
                if(ch) // add(trans[ch].fail,ch);// 构建失配树 fail指针向自己连边
                    trans[ch].fail=trans[trans[u].fail].ch[i],q.push(ch);
                else// 如果没有儿子 那么将fail的儿子作为儿子
                    ch=trans[trans[u].fail].ch[i];  
            }
        }
    }
    // t在AC自动机中跑一遍
    void find (char *t) { 
        int n = strlen(t + 1);
        for (int i = 1, p = 0; i <= n; i++) {
            int c = t[i] - 'a';
            p = trans[p].ch[c];
            // ......
        }
    }
}
\end{minted}

字符集较大的情况:可以用map来维护 tran[N][26] ,也可以用可持久化数组维护map<int,int> trans[N]

\begin{minted}{c++}
vector<pair<int,int>> trie[N]; // trie树的结构,邻接表vector
map<int,int> trans[N];
int fail[N];
void getfail()
{
    queue<int>q;
    for(auto it:trie[0]) {
        int u=0,v=it.second,to=it.first;
        // u-(to)-> v
        trans[u][to]=v;
        fail[v]=u;
        q.push(v);
    }
    while(!q.empty()) {
        int u=q.front(); q.pop();
        trans[u]=trans[fail[u]];//先把fail的trans 复制移过来
        
        for(auto it:trie[u]) {
            int v=it.second,to=it.first;
            trans[u][to]=v; // 然后修改
            fail[v]=trans[fail[u]][to]; 
            q.push(v);
        }
    }
}
// 可持久花数组维护map<int,int> trans[N]
struct Node {
    int l,r,v;
}trans[N*40];// 1~n trans[u][1~n] = v
int rt,cnt;
\end{minted}
给定 $n$ 个字符串 $s_{1 \dots n}$。, $q$ 次询问 $s_{l \dots r}$ 在 $s_k$ 中出现了多少次。$n,q,\sum_{i=1}^n |s_i| \le 10^5$。
\begin{tcolorbox}

\par \noindent 出现多串一定要考虑一下根号分治。

\begin{itemize}
\item 对于 $|s_k| < B$ 的,可以在 ac 自动机上暴力跑,差分一下用扫描线求一下,需要一个 $O(\sqrt N) - O(1)$ 的数据结构。
\item 对于 $|s_k| \geq B$ 的,这样的串不会有很多种,对于每种,把该串在ac自动机上的路径加1,然后计算子树和就可获得每个点在该串的出现次数。
\end{itemize}
\par \noindent 总复杂度 $O(n\sqrt n)$

\end{tcolorbox}

\begin{minted}{c++}
// 1. s[k]在 s[l~r]出现多少次
// 2. s[l~r]在 s[k] 中共出现多少次

// 如何理解 s[k] 在 s[r] 中出现的次数

// 1. 修改:s[r]的子串是根到 pos[r] 的链 + 1
//    查询:pos[k]子树即可               差分 + 扫描线

// 2. 修改:将 pos[k] 的子树打上标记 
//    查询:根到 pos[r] 的链上有多少标记 差分 + 扫描线 

#include <bits/stdc++.h>
using namespace std;


using LL = long long;
using PII = pair<int, int>;

const  int N = 1000005;
struct ACAM{
  int ch[N][26], cnt = 1, q[N], fa[N], head[N], ne[N];
  int ins(int p, int x) {
    if (!ch[p][x]) ch[p][x] = ++cnt;
    return ch[p][x];
  }
  void build() {
    int l, r;
    q[l = r = 1] = 1;
    while (l <= r) {
      int x = q[l++];
      if (x > 1 && !fa[x]) fa[x] = 1;
      for (int i = 0; i < 26; i++) {
        int &y = ch[x][i], z = ch[fa[x]][i];
        if (y) {
          q[++r] = y, fa[y] = z;
        } else {
          y = z;
        }
      }
    }

    for (int i = 2; i <= r; i++) {
      int x = fa[q[i]];
      assert(x);
      ne[i] = head[x], head[x] = i;
    }
    dfs(1);
  }

  int in[N], out[N], dfn;
  void dfs(int x) {
    in[x] = dfn++;
    for (int i = head[x]; i; i = ne[i]) {
      dfs(q[i]);
    }
    out[x] = dfn;
  }

  int val[N];
  void clear() {
    memset(val, 0, cnt + 1 << 2);
  }
  void cal() {
    for (int i = cnt; i; i--) {
      val[fa[q[i]]] += val[q[i]];
    }
  }
} ac;

template <class T>
struct DS {
  T s0[N], s1[(N >> 8) + 1];
  void add(int l, int r, const T &x) {
    while (l & 255 && l < r) s0[l++] += x;
    while (l + 256 <= r) s1[l >> 8] += x, l += 256;
    while (l < r) s0[l++] += x;
  }
  T ask(int p) {
    return s0[p] + s1[p >> 8];
  }
};

DS<int> ds;

int n, m, pos[N];
string s[N];
int main() {
  
  ios::sync_with_stdio(false);
  cin.tie(nullptr);
  cin >> n >> m;
  for (int i = 1; i <= n; i++) {
    cin >> s[i];
    int p = 1;
    for (char c : s[i]) {
      p = ac.ins(p, c - 'a');
    }
    pos[i] = p;
  }

  vector<LL> ans(m);
  vector<array<int, 4>> q0, q1;
  for (int i = 0, l, r, k; i < m; i++) {
    cin >> l >> r >> k;
    if (s[k].size() < 350) {
      q0.push_back({r, 1, k, i});
      q0.push_back({l - 1, -1, k, i});
    } else {
      q1.push_back({k, l, r, i});
    }
  }

  sort(q0.begin(), q0.end());
  sort(q1.begin(), q1.end());

  ac.build();
  
  for (int i = 0, j = 1; i < q0.size(); i++) {
    while (j <= q0[i][0]) {
      int x = pos[j++];
      ds.add(ac.in[x], ac.out[x], 1);
    }
    int p = 1;
    LL res = 0;
    for (char c : s[q0[i][2]]) {
      p = ac.ch[p][c - 'a'];
      res += ds.ask(ac.in[p]);
    }
    ans[q0[i][3]] += q0[i][1] * res;
  }

  vector<LL> sum(n + 1);
  for (int i = 0; i < q1.size(); i++) {
    auto &[k, l, r, id] = q1[i];
    if (!i || q1[i - 1][0] != k) {
      ac.clear();
      int p = 1;
      for (char c : s[k]) {
        p = ac.ch[p][c - 'a'];
        ac.val[p]++;
      }
      
      ac.cal();     // 子树求和
      // ac.val[pos[j]] 是 s[j] 在 s[k] 出现的次数
      for (int j = 1; j <= n; j++) {
        sum[j] = sum[j - 1] + ac.val[pos[j]];
      }
    }
    ans[id] = sum[r] - sum[l - 1];
  }

  for (auto z : ans) cout << z << "\n";
  return 0;
}
\end{minted}