\section{字典树}
\subsection{普通Trie}
\begin{minted}{c++}
struct Trie {
    int ch[maxn][30];
    bool has[maxn], vis[maxn];
    int cnt;
    void init() {
        memset(ch, 0, sizeof ch);
        memset(has, 0, sizeof has);
        memset(vis, 0, sizeof vis);
        cnt = 1;
    }
    void insert(char *s, int n) {
        int u = 0;
        for (int i = 0; s[i]; i++) {
            int c = s[i] - 'a';
            if (!ch[u][c]) ch[u][c] = ++cnt;
            u = ch[u][c];
        }
        has[u] = 1;
    }
    int query(char *s, int n) {
        int u = 0;
        for (int i = 0; i < n; i++) {
            int c = s[i] - 'a';
            if (!ch[u][c]) return -1;
            u = ch[u][c];
        }
        if (!has[u]) return -1;
        if (!vis[u]) {
            vis[u] = 1;
            return 0;
        } else
            return 1;
    }
};
\end{minted}
对于 0-1 字典树:
\begin{itemize}
\item 找异或最大值:当前位是 1 就走 0,是 0 就走 1,;走不通再走另一个;
\item 找与/或的最大值:以与运算为例,如果当前位是 1,那么肯定优先走 1;如果当前位是 0,那么当前位 和 0 或 1 运算的结果都是 0,我们无法确定走哪条支路才是最优解。于是我们可以将两条路合并成一条,把 1 的树自底向上合并到 0 的树
\end{itemize}

\subsection{压位Trie}
\begin{minted}{c++}
namespace BITWISE{
    inline int clz(unsigned long long x){return __builtin_clzll(x);}//这个函数是查询开头几个零
    inline int ctz(unsigned long long x){return __builtin_ctzll(x);}//这个函数是查询末尾几个零
} // namespace BITWISE

using namespace BITWISE;

typedef unsigned long long ull;
const int g = 6;
const int mod = (1 << g) - 1;
ull BUFF[1 << 25], *BT = BUFF + sizeof(BUFF) / sizeof(ull);//预先开好内存池
inline ull *alloc(int sz){return BT -= sz;}//动态分配空间
struct Trie{
    int dep;ull *a[5];//动态数组
    Trie(int sz){//初始化
        dep = 1;
        for(;;++ dep){
            int cnt = (sz + (1ull << g * dep) - 1) >> g * dep;//表示这一层有多少个点
            a[dep - 1] = alloc(cnt);
            if(cnt == 1) return ;
        }
        //注意这里层数越小越深,这样方便我们位运算
    }
    inline void ins(int x){
        for (int i = 0;i < dep;++ i){//自下而上遍历的
            ull p = 1ull << (x >> i * g & mod); //判断我们这个 x 在当前这一层要走哪一条边,并且直接左移好方便压位的处理
            if(a[i][x >> (i + 1) * g] & p) return ;//剪枝,上面有就可以弹出了
            a[i][x >> (i + 1) * g] |= p;
        }
    }
    inline void del(int x){
        for (int i = 0;i < dep;++ i)
            if(a[i][x >> (i + 1) * g] &= ~(1ull << (x >> i * g & mod))) return ;//删除一个位置,同样是删完还有就不删了的剪枝
    }
    inline int succ(int x){
        for (int i = 0;i < dep;++ i){
            int cur = (x >> i * g) & mod;ull v = a[i][x >> (i + 1) * g];//当前是哪一条边,由于这里只需要知道是哪一条边所以我们不需要左移
            if(v >> cur > 1){//如果存在前驱,也可以写成 v >> (cur + 1),后者更好理解但前者似乎更快
                int res = x >> (i + 1) * g << (i + 1) * g;
                res += (ctz(v >> (cur + 1)) + cur + 1) << i * g;//先把这一层的贡献加上,注意是不完整的
                for (int j = i - 1;~j;-- j) res += ctz(a[j][res >> (j + 1) * g]) << j * g;//剩下每一层都是完整的
                return res;//直接返回
            }
        }
        return 0;//否则返回零
    }
    inline int pre(int x){//与上面同理,不赘述
        for (int i = 0;i < dep;++ i){
            int cur = (x >> i * g) & mod;ull v = a[i][x >> (i + 1) * g];
            if(v & ((1ull << cur) - 1)){
                int res = x >> (i + 1) * g << (i + 1) * g;
                res += (mod - clz(v & ((1ull << cur) - 1))) << i * g;
                for (int j = i - 1;~j;-- j) res += (mod - clz(a[j][res >> (j + 1) * g])) << j * g;
                return res;
            }
        }
        return 0;
    }

};
Trie s(1 << 30);

void work(){
    /**
    1.插入 x 数(若已有 x 则不进行此操作);
    2.删除 x 数(若 x 不存在则不进行此操作);
    3.求 x 的前趋(前趋定义为 小于 x,且最大的数,若不存在则答案为 0);
    4.求 x 的后继(后继定义为 大于 x,且最小的数,若不存在则答案为 0);
    **/
    int op, x, ans;

      if(op == 0) s.ins(x);
    else if(op == 1) s.del(x);
    else if(op == 2) ans = s.pre(x);
    else ans = s.succ(x);
}
\end{minted}